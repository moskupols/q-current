\documentclass[oneside,russian, a4paper]{amsart}
\usepackage[T2A]{fontenc}
\usepackage[utf8]{inputenc}
\usepackage{fancyhdr}
\usepackage[margin=1in]{geometry}
\pagestyle{fancy}
%\setlength{\parskip}{\medskipamount}
%\setlength{\parindent}{0pt}
\usepackage[russian]{babel}
\usepackage{amsthm}
\usepackage{amstext}
\usepackage{amssymb}
\usepackage{cmap}
\usepackage[unicode=true,
 bookmarks=true,bookmarksnumbered=false,bookmarksopen=false,
 breaklinks=false,pdfborder={0 0 1},backref=false,colorlinks=false]
 {hyperref}
\usepackage{braket} % for \braket, \set and \Set

\makeatletter

\DeclareRobustCommand{\cyrtext}{
  \fontencoding{T2A}\selectfont\def\encodingdefault{T2A}}
\DeclareRobustCommand{\textcyr}[1]{\leavevmode{\cyrtext #1}}
\AtBeginDocument{\DeclareFontEncoding{T2A}{}{}}

%%%%%%%%%%%%%%%%%%%%%%%%%%%%%% Textclass specific LaTeX commands.
%\numberwithin{equation}{section}
%\numberwithin{figure}{section}
 \theoremstyle{definition}
 \newtheorem*{define*}{\protect\definitionname}
 \theoremstyle{plain}
 \newtheorem{Th}{\protect\theoremname}
 \theoremstyle{remark}
 \newtheorem{remark}{\protect\remarkname}
 \theoremstyle{remark}
 \newtheorem*{remark*}{\protect\remarkname}
 \theoremstyle{definition}
 \newtheorem{exercise}{\protect\exercisename}
 \theoremstyle{remark}
 \newtheorem{claim}{\protect\claimname}
 \theoremstyle{remark}
 \newtheorem*{claim*}{\protect\claimname}
 \theoremstyle{definition}
 \newtheorem{example}{\protect\examplename}
 \theoremstyle{plain}
 \newtheorem{lem}{\protect\lemmaname}
 \theoremstyle{plain}
 \newtheorem*{algorithm*}{\protect\algorithmname}

\makeatother
  \providecommand{\claimname}{Утверждение}
  \providecommand{\definitionname}{Определение}
  \providecommand{\examplename}{Пример}
  \providecommand{\exercisename}{Упражнение}
  \providecommand{\lemmaname}{Лемма}
  \providecommand{\remarkname}{Замечание}
  \providecommand{\theoremname}{Теорема}
  \providecommand{\algorithmname}{Алгоритм}

  \renewcommand\le{\leqslant}
  \renewcommand\ge{\geqslant}


\newcommand{\raigorname}[1]{\renewcommand{\raigor}{#1}}
\newcommand{\raigoraction}[1]{\renewcommand{\raigortold}{#1}}
\providecommand\raigor{Райгор}
\providecommand\raigortold{читал}
\providecommand{\gitlink}{Конспект лекций, которые \raigor{} \raigortold{} на ФИВТ МФТИ осенью 2014 года.
Исходники лежат, благодарности полнятся, а багрепорты принимаются по адресу \url{https://github.com/moskupols/q-current}.}

%\newcommand{\relmid}{\mathrel{}|\mathrel{}}
\newcommand{\abs}[1]{\left\lvert #1 \right\rvert}



\begin{document}

\title{Discrete Analysis, 3rd semester}

\author{Фёдор Алексеев}

\maketitle
\tableofcontents\newpage{}

\part{Лекция 1. Оценки и асимпотики комбинаторных величин}

\begin{example} Рассмотрим задачу.

  Пусть $n = 4k, k \in \mathbb{N}$. Рассмотрим граф 
  \begin{align*}
	G = (V,E)\mbox{: } &V = \left\{A \subset \left\{1, 2, \ldots, n\right\}| |A| = 2k\right\}, \\
	&E = \left\{ \left\{ A,B \right\}| |A \cap B| = k \right\}.
  \end{align*}

  Факты:
  \begin{enumerate}
	\item Сколько вершин в $G$? $|V| = C_n^{n/2}$
	\item Сколько рёбер? $|E| = C_n^{n/2} \cdot \left( C_{n/2}^{n/4} \right)^2 \cdot \frac{1}{2}$
	\item Сколько треугольников? 
	  \begin{align*}
		&\frac{1}{6} \cdot C_n^{n/2} \cdot \left( C_{n/2}^{n/4} \right)^2 \cdot 
		\sum_{i=0}^{n/4}\left( C_{n/4}^{i} \right)^2 \cdot \left( C_{n/4}^{n/4-i} \right)^2 = \\
		&\frac{1}{6} \cdot C_n^{n/2} \cdot \left( C_{n/2}^{n/4} \right)^2 \cdot \sum_{i=0}^{n/4}\left( C_{n/4}^{i} \right)^4
	  \end{align*}
  \end{enumerate}
\end{example}

\begin{claim}
  \(
	C_{n}^{0} < C_{n}^{1} < C_{n}^{2} < \ldots < C_{n}^{n/2} = C_{n}^{n/2} > \ldots
  \).
\end{claim}

\begin{claim}
  $C_{2n}^{n} < 2^{2n}$.
\end{claim}

\begin{claim}
  $C_{2n}^{n} > \frac{2^{2n}}{2n+1}$.
\end{claim}

\begin{remark*}
  Оба факта показываются из того, что $\sum_{i=0}^{2n}C_{2n}^{i} = 2^{2n}$.
\end{remark*}

\begin{define*}
  $a \sim b \Leftrightarrow \lim_{n\to\infty}\frac{a}{b} = 1 \Leftrightarrow a = b + o(1)$.
\end{define*}

\begin{remark*}
  Верна формула Стирлинга: $\boxed{n! \sim \sqrt{2\pi n}\left( \frac{n}{e} \right)^n}$.
\end{remark*}

\begin{claim}
  $C_{2n}^{n} \sim \frac{2^{2n}}{\sqrt{\pi n}}$.
\end{claim}

\begin{remark*}
  Ещё одна аналитическая запись:
  \begin{align*}
	a > 1  \Rightarrow	f(n) = (a+o(1))^n \Rightarrow \ln{f(n)} = n\ln{a+o(1)} \sim n\ln{a} \\
	(\nRightarrow f(n) \sim a^n).
  \end{align*}
  Тогда
  $
  \left.
  \begin{aligned}
	\ln(C_{2n}^{n}) &<&2n\ln2 &\\
	\ln(C_{2n}^{n}) &>&2n\ln2 &- \ln(2n+1) \sim 2n\ln2
  \end{aligned}
  \right|
  \Rightarrow \ln(C_{2n}^{n}) \sim 2n\ln2  \Rightarrow C_{2n}^{n} = (2 + o(1))^{2n} = (4 + o(1))^n
  $
\end{remark*}

\begin{Th}
  Пусть $a \in (0, 1)$. Тогда $C_{n}^{[an]} = \left( \frac{1}{a^a(1-a)^{1-a}} + o(1) \right)^n$.
\end{Th}
\begin{proof}
  \begin{remark*}
	При $a=\frac{1}{2}$ $C^{[n/2]}_{n} = \left( \frac1{\left(\frac12 \right)^{\frac12}\left( \frac12 \right)^\frac12} + o(1) \right)^n$.
  \end{remark*}
  \begin{remark*}
	$\frac1{a^a(1-a)^{1-a}} = e^{\overbrace{-a\ln{a} - (1-a)\ln(1-a)}^{\mbox{энтропия}}}$. Поэтому доказываем мы энтропийные оценки.
  \end{remark*}

  Перейдём к доказательству.
  $$
	C_{n}^{[an]} = \frac{n!}{[an]!(n-[an])!} \sim
	\frac{\sqrt{2\pi n}(\frac{n}{e})^n}{\sqrt{2\pi [an]}(\frac{[an]}{e})^{[an]} \sqrt{2\pi (n-[an])}(\frac{n-[an]}{e})^{n-[an]}} \\
  $$

  Допустим пока, будто нет \texttt{[]}~--- целых частей. Тогда получим:
  \begin{multline*}
	\underbrace{\frac{\sqrt{2\pi n}}{\sqrt{2\pi an}\sqrt{2\pi (n-an)}}}_{P(n)} \cdot 
	\left( \frac{n}{e} \right)^n \cdot \left( \frac{an}{e} \right)^{-an} \cdot \left( \frac{n-an}{e} \right)^{-n+an} = \\
	P(n) \cdot \frac{1}{a^{an}(1-a){n-an}} = P(n) \cdot \left( \frac{1}{a^a \cdot (1-a)^(1-a)} \right)^n =
	\left( \frac{1}{a^a \cdot (1-a)^{1-a}} + o(1) \right)^n
  \end{multline*}

  А почему можно избавиться от целых частей? Под корнем --- не важно, можно загнать в $P$. А вот $[an]^{[an]}$.
  \begin{align*}
	[an]^{[an]} = (an-\varepsilon)^{an-\varepsilon} = (an)^{an-\varepsilon}(1-\frac{\varepsilon}{an})^{an-\varepsilon} \sim
	(an)^{an}\underbrace{(an)^{-\varepsilon}e^{-\varepsilon}}_{\mbox{добавим в $P(n)$}}.
  \end{align*}
\end{proof}

\begin{exercise}
  Проверить для $(n-[an])^{n-[an]}$.
\end{exercise}

\begin{exercise} (Важное) \\
  Докажите, что $P([a_1n], [a_2n], [a_3n], \ldots, [a_kn]) = \left( \frac{(\sum{a_i})^{\sum{a_i}}}{\prod{a_i^{a_i}}} + o(1)\right)$.
\end{exercise}

\begin{exercise}
  Пусть $f(n) \sim an, a \in (0, 1)$. Тогда $C_{n}^{f(n)} = \left( \frac{1}{a^a(1-a)^{1-a}} + o(1) \right)^n$.
\end{exercise}

\begin{Th}
  \begin{enumerate}
	\item $C_{n}^{k} \le \frac{n^k}{k!}$
	\item $C_{n}^{k} \sim \frac{n^k}{k!}$, если $k^2 = o(n)$
	\item $C_{n}^{k} = \frac{n^k}{k!} * e^{-\frac{k(k-1)}{2n} + O(\frac{k^3}{r^2})}$
  \end{enumerate}
\end{Th}

\begin{proof}
  Для начала, $C_{n}^{k} = \frac{n(n-1)\cdots(n-k+1)}{k!} \le \frac{n^k}{k!}$. Чуть аккуратнее:
  \begin{multline*}
	C_{n}^{k} = \frac{n(n-1)\cdots(n-k+1)}{k!} = \frac{n^k}{k!}(1-\frac1n)(1-\frac2n)\cdots(1-\frac{k-1}{n}) = \\
	\frac{n^k}{k!}e^{\ln(1-\frac1n) + \ln(1-\frac2n) +\ldots+\ln(1-\frac{k-1}{n})} = 
	\frac{n^k}{k!}e^{-\frac1n + O(\frac1n) -\frac2n + O(\frac4{n^2}) + \ldots - \frac{k-1}{n} + O(\frac{(k-1)^2}{n^2})} = \\
	\frac{n^k}{k!}e^{-\frac{k(k-1)}{2n} + O(\frac{k^3}{n^2})}.
  \end{multline*}
\end{proof}

\end{document}

