\documentclass[oneside,russian, a4paper]{amsart}
\usepackage[T2A]{fontenc}
\usepackage[utf8]{inputenc}
\usepackage{fancyhdr}
\usepackage[margin=1in]{geometry}
\pagestyle{fancy}
%\setlength{\parskip}{\medskipamount}
%\setlength{\parindent}{0pt}
\usepackage[russian]{babel}
\usepackage{amsthm}
\usepackage{amstext}
\usepackage{amssymb}
\usepackage{cmap}
\usepackage[unicode=true,
 bookmarks=true,bookmarksnumbered=false,bookmarksopen=false,
 breaklinks=false,pdfborder={0 0 1},backref=false,colorlinks=false]
 {hyperref}
\usepackage{braket} % for \braket, \set and \Set

\makeatletter

\DeclareRobustCommand{\cyrtext}{
  \fontencoding{T2A}\selectfont\def\encodingdefault{T2A}}
\DeclareRobustCommand{\textcyr}[1]{\leavevmode{\cyrtext #1}}
\AtBeginDocument{\DeclareFontEncoding{T2A}{}{}}

%%%%%%%%%%%%%%%%%%%%%%%%%%%%%% Textclass specific LaTeX commands.
%\numberwithin{equation}{section}
%\numberwithin{figure}{section}
 \theoremstyle{definition}
 \newtheorem*{define*}{\protect\definitionname}
 \theoremstyle{plain}
 \newtheorem{Th}{\protect\theoremname}
 \theoremstyle{remark}
 \newtheorem{remark}{\protect\remarkname}
 \theoremstyle{remark}
 \newtheorem*{remark*}{\protect\remarkname}
 \theoremstyle{definition}
 \newtheorem{exercise}{\protect\exercisename}
 \theoremstyle{remark}
 \newtheorem{claim}{\protect\claimname}
 \theoremstyle{remark}
 \newtheorem*{claim*}{\protect\claimname}
 \theoremstyle{definition}
 \newtheorem{example}{\protect\examplename}
 \theoremstyle{plain}
 \newtheorem{lem}{\protect\lemmaname}
 \theoremstyle{plain}
 \newtheorem*{algorithm*}{\protect\algorithmname}

\makeatother
  \providecommand{\claimname}{Утверждение}
  \providecommand{\definitionname}{Определение}
  \providecommand{\examplename}{Пример}
  \providecommand{\exercisename}{Упражнение}
  \providecommand{\lemmaname}{Лемма}
  \providecommand{\remarkname}{Замечание}
  \providecommand{\theoremname}{Теорема}
  \providecommand{\algorithmname}{Алгоритм}

  \renewcommand\le{\leqslant}
  \renewcommand\ge{\geqslant}


\newcommand{\raigorname}[1]{\renewcommand{\raigor}{#1}}
\newcommand{\raigoraction}[1]{\renewcommand{\raigortold}{#1}}
\providecommand\raigor{Райгор}
\providecommand\raigortold{читал}
\providecommand{\gitlink}{Конспект лекций, которые \raigor{} \raigortold{} на ФИВТ МФТИ осенью 2014 года.
Исходники лежат, благодарности полнятся, а багрепорты принимаются по адресу \url{https://github.com/moskupols/q-current}.}

%\newcommand{\relmid}{\mathrel{}|\mathrel{}}
\newcommand{\abs}[1]{\left\lvert #1 \right\rvert}



\newenvironment{lyxcode}
{\par\begin{list}{}{
\setlength{\rightmargin}{\leftmargin}
\setlength{\listparindent}{0pt}% needed for AMS classes
\raggedright
\setlength{\itemsep}{0pt}
\setlength{\parsep}{0pt}
\normalfont\ttfamily}%
 \item[]}
{\end{list}}

\begin{document}

\title{$\mbox{Теоррия формальных языков}$}
\author{Фёдор Алексеев}
\raigorname{А.\,А.\,Сорокин}

\maketitle
\gitlink\tableofcontents\newpage{}

\part*{Лекция 1. Конечные автоматы}

\section{НКА}
\begin{define*}
\emph{Алфавит} --- непустое конечное множество символов $\Sigma$.
Множество \emph{слов} всех слов над данным алфавитом --- $\Sigma^{\star}$.
$\varepsilon\in\Sigma^{\star}$ --- \emph{пустое слово}. $L\subset\Sigma^{\star}$
--- \emph{формальный язык}.

Операция \emph{конкатенации} языков: $L_{1}\cdot L_{2}=\left\{ uv|u\in L_{1},v\in L_{2}\right\} $
(точку будем опускать).
\end{define*}

\begin{define*}
\emph{Конечный недетерменированный автомат }$M=\left\langle Q,\Sigma,\Delta,q_{0},F\right\rangle $,
где $Q$ --- множество \emph{состояний},$\Sigma$ --- \emph{алфавит},
$\Delta\subset Q\times\Sigma^{\star}\times Q$ --- конечное множество
\emph{переходов}, $q_{0}\in Q$ --- \emph{стартовое состояние}, $F\subset Q$
--- множество \emph{терминальных состояний}.
\end{define*}

\begin{define*}
Определим $\vdash$ --- наименьшее рефлексивное транзитивное отношение,
такое что

1) $\forall\left(\left\langle q_{1},w\right\rangle \mapsto q_{2}\right)\in\Delta\rightarrow\left\langle q_{1},w\right\rangle \vdash\left\langle q_{2},\varepsilon\right\rangle $

2) $\left\langle q_{1},u\right\rangle \vdash\left\langle q_{2},\varepsilon\right\rangle $,
$\left\langle q_{2},w\right\rangle \vdash\left\langle q_{3},\varepsilon\right\rangle \Rightarrow\left\langle q_{1},uw\right\rangle \vdash\left\langle q_{3},\varepsilon\right\rangle $

3) $\left\langle q_{1},u\right\rangle \vdash\left\langle q_{2},\varepsilon\right\rangle \Rightarrow\left\langle q_{1},uv\right\rangle \vdash\left\langle q_{2},v\right\rangle $
\end{define*}

\begin{define*}
Язык задаваемый автоматом $M$ --- $L(M)=\left\{ w\in\Sigma|\exists q\in F:\ \left\langle q_{0},w\right\rangle \vdash\left\langle q,\varepsilon\right\rangle \right\} $.
Такие языки называются \emph{автоматными}.\end{define*}
\begin{lem}
$\forall L$ --- автоматного языка $\exists M^{\prime}$ --- конечный
автомат, такой что $L(M^{\prime})=L$ и $\forall\left(\left\langle q_{1},w\right\rangle \mapsto q_{2}\right)\in\Delta^{\prime}\rightarrow\left|w\right|\le1$.\end{lem}
\begin{proof}
Построим автомат явно, добавив на каждом переходе $\left|w\right|-1$
новых состояний и сделав между ними однобуквенные переходы.\end{proof}
\begin{lem}
$\forall L$ --- автоматного языка $\exists M^{\prime}$ --- конечный
автомат, такой что все переходы имеют длину $\le1$ и $\left|F\right|=1$\end{lem}
\begin{proof}
Строим автомат $M^{\prime}=\left\langle Q\cup\left\{ q_{f}\right\} ,\Sigma,\Delta^{\prime},q_{0},\left\{ q_{f}\right\} \right\rangle $,
где $\Delta^{\prime}=\Delta\cup\left\{ \left\langle q,\varepsilon\right\rangle \mapsto q_{f}|q\in F\right\} $.\end{proof}
\begin{lem}
$\forall L$ --- автоматного языка $\exists M^{\prime}$ --- НКА с
однобуквенными переходами, его распознающий\end{lem}
\begin{proof}
Построим автомат $M^{\prime}=\left\langle Q,\Sigma,\Delta^{\prime},q_{0},F^{\prime}\right\rangle $.

$\Delta^{\prime}=\Delta\cup\left\{ \left(q_{1},a\right)\mapsto q_{2}|\exists q_{3}:\ \left(q_{1},\varepsilon\right)\vdash q_{3},\ \left(q_{3},a\right)\vdash q_{2}\right\} $.
$F^{\prime}=\left\{ q|\exists q^{\prime}\in F^{\prime}:\ \left(\left\langle q,\varepsilon\right\rangle \vdash\left\langle q^{\prime},\varepsilon\right\rangle \right)\right\} \cup F$.

Пусть $w\in L(M)\Leftrightarrow\exists q\in F\ \exists q_{1},\ldots,q_{n}\in Q:\ \left\langle q_{0},a_{1},\ldots,a_{n}\right\rangle \vdash\left\langle q_{1},a_{1},\ldots,a_{n}\right\rangle \vdash\ldots\vdash\left\langle q_{n},\varepsilon\right\rangle \vdash\left\langle q,\varepsilon\right\rangle $.
Тогда в новом автомате это слово будет принято, так как по построению
из состояния $q_{i}$ есть переход по символу $a_{i+1}$ в $q_{i+1}$,
а $q_{n}\in F^{\prime}$.\end{proof}
\begin{algorithm*}
Распознавания слова автоматом\end{algorithm*}
\begin{lyxcode}
  curQ~=~$\left\{ q_{0}\right\} $

  for~c~in~word:
	\begin{lyxcode}
	newQ~=~$\varnothing$

	for~state~in~curQ:
	  \begin{lyxcode}
	  for~$\left(state,c,newState\right)$~in~$\Delta$:
		\begin{lyxcode}
		newQ~=~newQ$\cup\left\{ newState\right\} $
		\end{lyxcode}
	  \end{lyxcode}
	curQ~=~newQ
	\end{lyxcode}
  if~curQ$\cap F=\varnothing$~then~accept~word

  else~discard~word
\end{lyxcode}

\section{ДКА}
\begin{define*}
$M=\left\langle Q,\Sigma,\Delta,q_{0},F\right\rangle $ --- ДКА, если

1) $\forall\left(\left\langle q_{1},a\right\rangle \mapsto q_{2}\right)\in\Delta\Rightarrow\left|a\right|=1$

2) $\forall q\in Q,\ a\in\Sigma\rightarrow\left|\left\{ q_{2}|\left\langle q_{1},a\right\rangle \mapsto q_{2}\right\} \right|\le1$\end{define*}
\begin{claim*}
$\forall q_{1}\in Q\ \forall w\in\Sigma^{\star}\rightarrow\left|\left\{ q_{2}|\left\langle q_{1},w\right\rangle \vdash\left\langle q_{2},\varepsilon\right\rangle \right\} \right|\le1$.\end{claim*}

\end{document}

