\documentclass[russian,twoside,a5paper]{amsart}
\usepackage{cmap}
\usepackage[T2A]{fontenc}
\usepackage[utf8]{inputenc}
\usepackage[margin=0.6in]{geometry}
\usepackage{amsthm}
\usepackage{amstext}
\usepackage{amssymb}
\usepackage[unicode=true,
 bookmarks=true,bookmarksnumbered=false,bookmarksopen=false,
 breaklinks=false,pdfborder={0 0 1},backref=false,colorlinks=false]
 {hyperref}
\usepackage{braket} % for \braket, \set and \Set
\usepackage[russian]{babel}
\usepackage{mathtext}

\usepackage{fancyhdr}
\setlength{\headheight}{15pt}

\pagestyle{fancy}

\fancyhf{}
\fancyhead[LE,RO]{\thepage}
\fancyhead[RE]{\textit{\nouppercase{\leftmark}}}
\fancyhead[LO]{\textit{\nouppercase{\rightmark}}}


\makeatletter

\DeclareRobustCommand{\cyrtext}{
  \fontencoding{T2A}\selectfont\def\encodingdefault{T2A}}
\DeclareRobustCommand{\textcyr}[1]{\leavevmode{\cyrtext #1}}
\AtBeginDocument{\DeclareFontEncoding{T2A}{}{}}

%%%%%%%%%%%%%%%%%%%%%%%%%%%%%% Textclass specific LaTeX commands.
%\numberwithin{equation}{section}
%\numberwithin{figure}{section}
 \theoremstyle{definition}
 \newtheorem*{define*}{\protect\definitionname}
 \theoremstyle{plain}
 \newtheorem{Th}{\protect\theoremname}
 \theoremstyle{remark}
 \newtheorem{remark}{\protect\remarkname}
 \theoremstyle{remark}
 \newtheorem*{remark*}{\protect\remarkname}
 \theoremstyle{definition}
 \newtheorem{exercise}{\protect\exercisename}
 \theoremstyle{remark}
 \newtheorem{claim}{\protect\claimname}
 \theoremstyle{remark}
 \newtheorem*{claim*}{\protect\claimname}
 \theoremstyle{definition}
 \newtheorem{example}{\protect\examplename}
 \theoremstyle{plain}
 \newtheorem{lem}{\protect\lemmaname}
 \theoremstyle{plain}
 \newtheorem*{algorithm*}{\protect\algorithmname}

\makeatother
  \providecommand{\claimname}{Утверждение}
  \providecommand{\definitionname}{Определение}
  \providecommand{\examplename}{Пример}
  \providecommand{\exercisename}{Упражнение}
  \providecommand{\lemmaname}{Лемма}
  \providecommand{\remarkname}{Замечание}
  \providecommand{\theoremname}{Теорема}
  \providecommand{\algorithmname}{Алгоритм}

  \renewcommand\le{\leqslant}
  \renewcommand\ge{\geqslant}


\newcommand{\raigorname}[1]{\renewcommand{\raigor}{#1}}
\newcommand{\raigoraction}[1]{\renewcommand{\raigortold}{#1}}
\providecommand\raigor{Райгор}
\providecommand\raigortold{читал}
\providecommand{\gitlink}{Конспект лекций, которые \raigor{} \raigortold{} на ФИВТ МФТИ весной 2015 года.
Исходники лежат, благодарности полнятся, а багрепорты принимаются по адресу \url{https://github.com/moskupols/q-current}.}

%\newcommand{\relmid}{\mathrel{}|\mathrel{}}
\newcommand{\abs}[1]{\left\lvert #1 \right\rvert}

\DeclareMathOperator{\ord}{ord}



\begin{document}

\title{$\mbox{Дифференциальные уравнения}$}
\author{Фёдор Алексеев}
\raigorname{В.\,Ю.\,Дубинская}
\raigoraction{читала}

\maketitle\gitlink\tableofcontents
%\newpage{}

\part*{Лекция 1. Основные понятия и методы решения некоторых уравнений первого
порядка.}

\section{Основные определения}
\begin{remark*}
Рассмотрим функцию действительного переменного $y(x)$. Мы обозначаем
$y=\frac{dy}{dx},\ldots,y^{(n)}=\frac{d^{n}y}{dx^{n}}$\end{remark*}
\begin{define*}
Уравнение вида $F(x,y,y^{\prime},\ldots,y^{(n)})=0$ называется \emph{обыкновенным
дифференциальным уравнением} (ОДУ) порядка $n$.
\end{define*}

\begin{define*}
Рассмотрим промежуток $I\subset\mathbb{R}$. Функция $\varphi(x)$,
зависящая от $x$ и определенная на $I$ называется \emph{решением}
дифференциального уравнения на $I$, если

(a) $\varphi(x)$ определена и непрерывна на $I$ со всеми своими
производными до порядка $n$

(b) $F(x,\varphi(x),\varphi^{\prime}(x),\ldots,\varphi^{(n)}(x))\equiv0$
на $I$.\end{define*}
\begin{remark*}
Промежутком мы называем в том числе и отрезок, и интервал и открытые
и закрытый луч.
\end{remark*}

\begin{define*}
График функции $\varphi(x)$, являющейся решением уравнения, называется
\emph{интегральной кривой} уравнения.
\end{define*}

\begin{remark*}
Уравнение $F(x,y,y^{\prime})=0$ --- ОДУ первого порядка.\end{remark*}
\begin{define*}
Если уравнение имеет вид $y^{\prime}=f(x,y)$, то говорят, что оно
\emph{разрешено относительно производной}.
\end{define*}

\section{Векторное поле направлений }
\begin{define*}
Рассмотрим уравнение, разрешенное относительно производной, где $f(x,y)$
определена в некоторой области $G\subset\mathbb{R}^{2}$. Тогда можно
представить геометрический смысл уравнения, разрешенного относительно
производной. В каждой точке области мы знаем угловой коэффициент касательной
к интегральной кривой в данной точке. Совокупность касательных векторов
в области $G$ называется\emph{ векторным полем направлений} уравнения
первого порядка. \end{define*}
\begin{remark*}
Любая интегральная кривая должна быть огибающей к этому полю.
\end{remark*}

\begin{define*}
ГМТ $f(x,y)=c,\ c\in\mathbb{R}$ называется \emph{изоклиной}.\end{define*}
\begin{example}
$y^{\prime}=y+x$. Рассмотрев изоклины $x+y=0$, $x+y=-1$, $x+y=1$
можно увидеть, что в области $x+y>0$ интегральные линии идут вверх,
в области $x+y<0$ --- вниз. При пересечении с $x+y=0$ наблюдается
экстремум (минимум). Продифференцировав $y^{\prime\prime}=y^{\prime}+1$,
мы видим, что, скорее всего все решения стремятся к прямой $x+y=-1$
асимптотическим образом. Такой анализ позволяет нестрого прикинуть
характер решений дифференциального уравнения.
\end{example}

\section{Общее решение ОДУ}
\begin{define*}
Функция $\varphi(x,c)$ --- где $c$ --- действительный параметр,
называется \emph{общим решением} ОДУ первого порядка, если

(a) при каждом допустимом значении $c$ функция $\varphi(x,c)$ является
решением уравнения

(b) для каждого решения уравнения найдется значение $c$ такое, что
указанное решение представимо в виде $y=\varphi(x,c)$.\end{define*}
\begin{remark*}
Определение, конечно, не гарантирует существование решения в таком
виде у всякого уравнения, однако, есть основания полагать, что оно
все-таки существует. Причиной тому будет то, что оператор дифференцирования
представляется собой линейное преобразование пространства дифференцируемых
на промежутке функций. Тогда решение дифференциального уравнение тесно
связано с отысканием ядра этого преобразования, а общее решение будет
восстановлено в виде общего решения однородной системы плюс частное
решение.
\end{remark*}

\begin{remark*}
Решение уравнения часто можно представить лишь в виде $\Phi(x,y)=0$,
которое задает неявную зависимость между $x$ и $y$. Такую запись
решения уравнения мы также будем считать корректной.\end{remark*}
\begin{define*}
\emph{Уравнением в дифференциалах} называется уравнение в области
$G$ 
\[
M(x,y)\cdot dx+N(x,y)\cdot dy=0,\ \left(x,y\right)\in G\Rightarrow M^{2}(x,y)+N^{2}(x,y)\ne0
\]
\end{define*}
\begin{remark*}
Уравнение в дифференциалах можно переписать в привычном виде, однако
их достоинство в том, можно легко рассматривать функции $x=\varphi(y)$,
которые также могут быть решениями уравнения.
\end{remark*}

\section{Задача Коши}
\begin{define*}
\emph{Задача Коши} для ОДУ первого порядка в одной из двух записей
(при том, что задана $\left(x_{0},y_{0}\right)\in G$) состоит в нахождении
решения уравнения, при котором интегральная кривая проходит через
точку $\left(x_{0},y_{0}\right)$.\end{define*}
\begin{Th}
Пусть в области $G$ определены функции $f(x,y)$ и $\frac{\partial f}{\partial x}(x,y)$
и пусть $\left(x_{0},y_{0}\right)\in G$, тогда

(a) $\exists$решение уравнения $y^{\prime}=f(x,y)$ такое, что $y(x_{0})=y_{0}$

(b) Если $y_{1}(x)$ и $y_{2}(x)$ --- решения уравнения (3), удовлетворяющие
условию $y(x_{0})=y_{0}$, то на пересечении промежутков их определения
они тождественно равны.
\end{Th}

\section{ОДУ с разделяющимися переменными}
\begin{define*}
\emph{Уравнением с разделяющимися переменными} называется уравнение
одного из двух видов: 
\[
y^{\prime}=f(x)\cdot g(y)
\]
\[
M(x)\cdot N(y)\cdot dx+P(x)\cdot Q(y)\cdot dy=0
\]
\end{define*}
\begin{remark*}
В данном случае область $G=D(f)\times D(g)$, где $D(f)$ --- область
определения функции $f$.\end{remark*}
\begin{algorithm*}
Решим это уравнение. Если $g(y)=0$, то мы получаем функции вида $y=c_{1},\ldots,y=c_{n}$
и уравнение вырождается в тождество. Значит это решения.

Иначе 
\[
\frac{y^{\prime}}{g(y)}=f(x)\Rightarrow\int\frac{dy}{g(y)}=\int f(x)\cdot dx+c\Rightarrow H(y)=F(x)+c\Rightarrow H^{-1}(F(x)+c)
\]
Обратная функция будет существовать, так как мы исключили случаи $g(y)=0$,
то есть $g$ --- знакопостоянна, значит $H(y)$ --- монотонна.\end{algorithm*}
\begin{exercise}
Решить уравнения и проверить теорему о существовании и единственности

\begin{gather*}
(1)\ y\cdot dx+x\cdot dx=0\\
(2)\ y^{\prime}=y\\
(3)\ y^{\prime}=3y^{2/3}
\end{gather*}
\end{exercise}
\begin{remark*}
У последнего уравнения наблюдается <<нарушение>> теоремы о существовании
и единственности, так как нет непрерывности производной.\end{remark*}


\end{document}

