\documentclass[russian,twoside,a5paper]{amsart}
\usepackage{cmap}
\usepackage[T2A]{fontenc}
\usepackage[utf8]{inputenc}
\usepackage[margin=0.6in]{geometry}
\usepackage{amsthm}
\usepackage{amstext}
\usepackage{amssymb}
\usepackage[unicode=true,
 bookmarks=true,bookmarksnumbered=false,bookmarksopen=false,
 breaklinks=false,pdfborder={0 0 1},backref=false,colorlinks=false]
 {hyperref}
\usepackage{braket} % for \braket, \set and \Set
\usepackage[russian]{babel}
\usepackage{mathtext}

\usepackage{fancyhdr}
\setlength{\headheight}{15pt}

\pagestyle{fancy}

\fancyhf{}
\fancyhead[LE,RO]{\thepage}
\fancyhead[RE]{\textit{\nouppercase{\leftmark}}}
\fancyhead[LO]{\textit{\nouppercase{\rightmark}}}


\makeatletter

\DeclareRobustCommand{\cyrtext}{
  \fontencoding{T2A}\selectfont\def\encodingdefault{T2A}}
\DeclareRobustCommand{\textcyr}[1]{\leavevmode{\cyrtext #1}}
\AtBeginDocument{\DeclareFontEncoding{T2A}{}{}}

%%%%%%%%%%%%%%%%%%%%%%%%%%%%%% Textclass specific LaTeX commands.
%\numberwithin{equation}{section}
%\numberwithin{figure}{section}
 \theoremstyle{definition}
 \newtheorem*{define*}{\protect\definitionname}
 \theoremstyle{plain}
 \newtheorem{Th}{\protect\theoremname}
 \theoremstyle{remark}
 \newtheorem{remark}{\protect\remarkname}
 \theoremstyle{remark}
 \newtheorem*{remark*}{\protect\remarkname}
 \theoremstyle{definition}
 \newtheorem{exercise}{\protect\exercisename}
 \theoremstyle{remark}
 \newtheorem{claim}{\protect\claimname}
 \theoremstyle{remark}
 \newtheorem*{claim*}{\protect\claimname}
 \theoremstyle{definition}
 \newtheorem{example}{\protect\examplename}
 \theoremstyle{plain}
 \newtheorem{lem}{\protect\lemmaname}
 \theoremstyle{plain}
 \newtheorem*{algorithm*}{\protect\algorithmname}

\makeatother
  \providecommand{\claimname}{Утверждение}
  \providecommand{\definitionname}{Определение}
  \providecommand{\examplename}{Пример}
  \providecommand{\exercisename}{Упражнение}
  \providecommand{\lemmaname}{Лемма}
  \providecommand{\remarkname}{Замечание}
  \providecommand{\theoremname}{Теорема}
  \providecommand{\algorithmname}{Алгоритм}

  \renewcommand\le{\leqslant}
  \renewcommand\ge{\geqslant}


\newcommand{\raigorname}[1]{\renewcommand{\raigor}{#1}}
\newcommand{\raigoraction}[1]{\renewcommand{\raigortold}{#1}}
\providecommand\raigor{Райгор}
\providecommand\raigortold{читал}
\providecommand{\gitlink}{Конспект лекций, которые \raigor{} \raigortold{} на ФИВТ МФТИ весной 2015 года.
Исходники лежат, благодарности полнятся, а багрепорты принимаются по адресу \url{https://github.com/moskupols/q-current}.}

%\newcommand{\relmid}{\mathrel{}|\mathrel{}}
\newcommand{\abs}[1]{\left\lvert #1 \right\rvert}

\DeclareMathOperator{\ord}{ord}



\renewcommand{\NOK}{\mathop{\text{НОК}}\nolimits}

\begin{document}

\title{$\mbox{Теория групп}$}
\author{Фёдор Алексеев}
\raigorname{И.\,И.\,Богданов}

\maketitle
\gitlink{}
\tableofcontents\newpage{}

\part*{Лекция 1. Группа, подгруппа. Изоморфизм групп.}
\section{Группа}
\begin{define*}
  \emph{Группа}~--- это множество $G$ с определённой на нём бинарной операцией (т.\,е. $\forall a, b \in G$ определён результат операции $a\cdot b \in G$)
  \begin{description}
	\item[1. Ассоциативность] $\forall a, b, c \in G \to (a\cdot b)\cdot c = a \cdot (b \cdot c)$;
	\item[2. Наличие нейтрального элемента] $\exists e \in G: \forall a \in G \to a\cdot e = e\cdot a$;
	\item[3. Наличие обратного элемента] $\forall g \in G \to \exists g^{-1} \in G: g\cdot g^{-1} = g^{-1}\cdot g = e$.
  \end{description}
  Группа \emph{абелева}(\emph{коммутативная}), если $\forall a, b \in G \to a\cdot b = b\cdot a$.
\end{define*}

\begin{remark*}
  Произведение $g_1 \cdot g_2 \cdot \cdots \cdot g_n$ также не зависит от порядка.
\end{remark*}

\begin{claim}
  $\forall g \in G \to \exists! g^{-1}$.
\end{claim}

\begin{proof}
  Пусть $a$~--- левый обратный к $g$ (т.\,е. $ag = e$),
  $b$~--- правый обратный к $g$ (т.\,е. $gb = e$). Докажем, что $a=b$:
  \begin{equation*}
	b = eb = (ag)b = a(gb) = ae = a
  \end{equation*}
\end{proof}

\begin{exercise}
  Условие 3 из определения можно заменить на $\forall g \in G \to \exists g^{-1}$~--- правый обратный.
\end{exercise}

\begin{example}
	$(\mathbb{Z}, +)$
\end{example}

\begin{example}
  $(\mathbb{R}, +)$, и даже $(F, +)$.
\end{example}

\begin{example}
  $(\mathbb{R}\setminus\{0\}, 0)$, и даже $F^* = (F\setminus\{0\}, \cdot)$.
\end{example}

\begin{remark*}
  Более общо, если $R$~--- кольцо, то $(R, +)$ и $(R^*, \cdot)$ --- абелевы группы ($R^*$ --- множество обратимых элементов $R$).
  Хорошо бы доказать, впрочем, что если $a, b \in R^*$, то $ab \in R^*$:
  \begin{equation*}
	ab(ab)^{-1} = abb^{-1}a^{-1} = aa^{-1} = e
  \end{equation*}
\end{remark*}

\begin{example}
  $M_{n\times n}(F)$ --- кольцо. $(M_{n\times n})^* = GL(F)$ --- мультипликативная группа невырожденных матриц $n\times n$.
\end{example}

\begin{example}
  $(\mathbb{Z}_n, +)$ и $(\mathbb{Z}_n^*, \cdot)$, где $\mathbb{Z}_n^*$ --- все взаимно простые с $n$.
\end{example}

\begin{proof}
  Если $(a, n) \neq 1$, то $ka \not\equiv 1 \pmod{n}$. Если $(a, n) = 1$, то $\exists u, v: au + nv = a	\Rightarrow	au \equiv 1 \pmod{n}$
\end{proof}
Значит $\abs{\mathbb{Z}_n^*} = \varphi(n)$.

\begin{define*}
  \emph{Порядок группы} G --- это количество её элементов $\abs{G}$.
\end{define*}

\begin{example}
  Пусть $S_n$ --- все перестановки множества $[n] = \left\{ 1, \ldots, n \right\}$, т.\,е. биекции $[n] \to [n]$. Тогда $(S_n, \circ)$ --- группа.

  Если $\Omega$ --- произвольное множество, то аналогичное множество будем обозначать $S(\Omega)$. Это тоже группа.
\end{example}

\section{Подгруппы}

\begin{define*}
  Пусть $G$ --- группа, $\varnothing \neq H \subset G$. $H$ --- \emph{подгруппа} $G$, если $\forall a, b \in H \to ab \in H, a^{-1} \in H$. 
  Обозначается как $H \le G$.
\end{define*}

\begin{define*}
  $\left\{ e \right\} \le G$, $G \le G$ --- несобственные подгруппы.
\end{define*}

\begin{example}
  $D_n(F^*) \le GL_n(F)$, где $D_n(F)$ --- множество диагональных матриц над $F$.
\end{example}

\begin{example}
  $GL_n(F) \ge SL_N(F) = \Set{ A \in GL_n(F) | \det{A} = 1 }$.
\end{example}

\begin{exercise}
  $GL_n(F) \ge T_n(F)$ ($T$ --- верхнетреугольные. Здесь с не нулями на диагонали.).
\end{exercise}

\begin{example}
  $GL_n(R)	\ge O_n$ --- группа ортогональных матриц ($A^{-1} = A^T$).
\end{example}

\begin{example}
  $O_n \ge \mathcal{D}_n = \Set{ f \in O_n: f(P_n) = P_n }$ ($P_n$ --- это многоугольник) --- группа диэдра.
\end{example}

\begin{exercise}
  $\abs{\mathcal{D}_n} = 2n$ (по $n$ поворотов и симметрий).
\end{exercise}

\section{Изоморфизм}

\begin{define*}
  Пусть $G$ и $H$ --- две группы. $\varphi: G\to H$ --- \emph{изоморфизм}, если $\varphi$ --- биекция, и $\forall a, b \to$
  \begin{enumerate}
	\item $\varphi(ab) = \varphi(a)\varphi(b)$
	\item $\varphi(a^{-1}) = (\varphi(a))^{-1}$.
  \end{enumerate}
\end{define*}

\begin{define*}
  Две группы \emph{изоморфны}, если $\exists \varphi$ --- изоморфизм.
\end{define*}

\begin{example}
  $\mathcal{D}_3 \cong S_3$
\end{example}

\begin{example}
  $\mathbb{C}^* \ge \mathbb{C}_n = \Set{ z \in \mathbb{C}: z^n = 1 }$. 
  $\mathbb{Z}_n \cong \mathbb{C}_n \cong$ группе вращений правильного $n$-угольника.
\end{example}

\begin{exercise}
  Свойство 2 определения изоморфизма не нужно.
\end{exercise}

\begin{define*}
  $G$ --- группа, $M \subset G$ --- подмножество. \emph{Подгруппой}, порождённой множеством $M$ называется пересечение всех подгрупп в $G$ содержащих $M$:
  $$(M) = \bigcap_{H\le G, M\le H}H$$
\end{define*}

\begin{claim}
  Пересечение любого семейства подгрупп --- подгруппа. (в т.\,ч., $(M)$ --- подгруппа)
\end{claim}

\begin{proof}
  Пусть $K = \underset{H_i \le G, i \in I}{\bigcap}H_i$. Если $a, b \in K$, то 
  $a, b \in H_i \Rightarrow 
  \left\{ 
	\begin{aligned}
	  ab \in H_i &\Rightarrow& ab \in K\\ 
	  a^{-1} \in H_i &\Rightarrow& a^{-1} \in K
	\end{aligned}
  \right.$, кроме того, $e \in k$.
\end{proof}

\begin{claim}
  $(M) = \Set{ a_1, \ldots, a_k: \forall i \le k \to a_i \in M \lor a^{-1} \in M }$.
\end{claim}

\begin{proof}
  Обозначим правую часть за $N$.

  \begin{itemize}
	\item $N \subset (M)$: если $a_i, \ldots, a_k \in N$, то $a_i \in (M) \Rightarrow a_1, \ldots, a_k \in (M)$.
	\item $(M) \subset N$: это так, ибо $N$ --- подгруппа, содержащая $M$.
		Если $m\in M\Rightarrow m\in N$. Если $a_{1},\ldots,a_{k},b_{1},\ldots,b_{l}\in N$,
		то $\left(a_{1}\cdot\ldots\cdot a_{k}\right)\left(b_{1}\cdot\ldots\cdot b_{l}\right)\in N$,
		то есть $N$ --- замкнута. $e\in N$ --- нейтральный, ассоциативность
		очевидна, $\left(a_{1}\cdot\ldots\cdot a_{k}\right)^{-1}=a_{k}^{-1}\cdot\ldots\cdot a_{1}^{-1}$.
  \end{itemize}
\end{proof}

\begin{example}
  $r\in D_{n}$ --- поворот на $\frac{2\pi}{n}$. Тогда $\left(r\right)\cong\mathbb{Z}_{n}$.
\end{example}


\part*{Лекция 2.}

Давайте перед тем, как пойдём дальше, предъявим ещё парочку примеров подгрупп, порождённых каким-то множеством.

\section{Циклические подгруппы}

\begin{example}
  \begin{enumerate}
	\item Каким наилучшим образом можно породить $\mathbb{Z}_n$?$\mathbb{Z}_n = (1)$.
	\item Пусть $(1) = H$. Верно ли, что $H = \mathbb{Z}$? Да, т.к. $1, 1+1, \cdots, -1, -1 - 1, \cdots$.
  \end{enumerate}
\end{example}

\begin{define*}
  Если $M \subset G$ таково, что $(M) = G$, то $F$ \emph{порождена} $M$, а $M$ --- \emph{порождающее множество} для $G$.
\end{define*}

\begin{example}
  $GL_n(F) = (\set{\mbox{элементарные матрицы}})$: любая невырожденная матрица раскладывается в произведение элементарных матриц.
\end{example}

\begin{define*}
  Группа $G$ называется \emph{циклической}, если она порождена одним элементом, т. е. $G = (g)$ для некоторого $g \in G$.
\end{define*}

\begin{remark*}
  В любой группе есть циклическая подгруппа (подгруппы).
\end{remark*}

\begin{Th}
  Если $G$ --- циклическая группа, то либо $G \cong \mathbb{Z}$, либо $G \cong \mathbb{Z}_n$.
  \begin{proof}
	Раз $G$ --- циклическая группа, $G = (g)$, $g \in G$. Рассмотрим все степени $g$. 

	А что такое степени, кстати? $g^0 = e$. Если $n > 0$, $g^n = g \cdot g \cdot g \cdot \ldots \cdot g$.
	Если $n < 0$, $g^n = g^{-1} \cdot \ldots \cdot g^{-1}$.

	\begin{exercise}[доказательство не закончилось, но это важный факт]
	  $g^{k+l} = g^k \cdot g^l$, $(g^k)^l = g^{kl}$.
	\end{exercise}

	\begin{claim}[Следствие]
	  $(g^k)^{-1} = g^{-k}$: $g^k \cdot g^{-k} = g^{k-k} = g^0 = e$.
	\end{claim}

	Итак, рассмотрим все степени $g$. Возможны два случая:
	\begin{enumerate}
	  \item Все выписанные степени $g$ различны. Однако мы выписали всю $G$, так как это циклическая группа. 
		Тогда рассмотрим $\varphi: \mathbb{Z} \to G, \varphi(k) = g^k$. 
		Это биекция, $\varphi(k+l)=\varphi(k)\cdot\varphi(l)$, $\varphi(-k)=\varphi(k)^-1$.
		Значит $\varphi$ --- изоморфизм, т.е. $g \cong \mathbb{Z}$.

	  \item $\exists k \neq l : g^k = g^l  \Rightarrow g^{k-l} = g^k\cdot (g^l)^{-1} = e = g^{l-k}$. Это значит, что $g$ в некоторой степени 
		суть единичный элемент: $\exists n \in \mathbb{N} : g^n = e$. Пусть мы выбрали наименьшее из всех возможных $n$.

		Тогда $g^0, g^1, \ldots, g^{n-1}$ попарно различны и образуют всю группу: $G = \set{g^0, \ldots, g^{n-1}}$.

		Пусть первое неверно, $g^k = g^l, 0 \le k < l < n \Rightarrow g^{l-k} = e \Rightarrow n$ не минимально.

		Пусть второе неверно, $\forall k \in \mathbb{Z} \to k = qn + r, q, r \in \mathbb{Z}, 0 \le r < n$.

		Наконец, пусть $\varphi: \mathbb{Z}_n \to G, \varphi(k) = g^k$. Вспомним теперь, что $\mathbb{Z}_n$ --- остатки от деления на $n$,
		тогда $\varphi$ --- биекция (по двум только что доказанным фактам), и $\varphi(k + l) = g^{k+l}$ либо $\varphi(k+l) = g^{k+l-n}$,
		но тогда в любом случае $g^{k+l} = \varphi(k) + \varphi(l)$, значит мы предъявили изоморфизм, значит $G \cong \mathbb{Z}_n$.
	\end{enumerate}
  \end{proof}
\end{Th}

\begin{remark*}
  Вообще, мы сейчас доказали, что $G \cong \mathbb{Z}_n$. Однако т.к. изоморфизм --- обратимая транзитивная операция, отношение изоморфности является отношением эквивалентности.
\end{remark*}

\section{Порядок элемента. Теорема Кэли}

\begin{define*}
  Пусть $G$ --- группа, $g \in G$. \emph{Порядок элемента} $g$ --- минимальное $n \in \mathbb{N}: g^n = e$. Если такого $n$ нет, то порядок $g$ --- бесконечность.

  \emph{Обозначение}: $\ord g, \abs{g}$.
\end{define*}

\begin{claim}
  $\abs{g} = \abs{(g)}$.
  \begin{proof} $ $

	\begin{itemize}
	  \item 
		Если $(g) \cong \mathbb{Z}$, то $\abs{g} = \infty$.
	  \item 
		Если $(g) \cong \mathbb{Z}_n$, то $\abs{g} = n$.
	\end{itemize}
  \end{proof}
\end{claim}

\begin{Th}
  Подгруппа циклической группы --- циклическая.

  \begin{proof}
	Можно считать, что $G = \mathbb{Z}$ или $G = \mathbb{Z}_n$.
	\begin{enumerate}
	  \item $G = \mathbb{Z}$. Пусть $H \subseteq \mathbb{Z}$. Если $H = (0)$, то очевидно. Иначе
		выберем $h \in H, h \neq 0$ с наименьшим $\abs{h}$. Тогда $H \ge (h) = h\mathbb{Z}$ --- подмножество всех элементов, кратных $h$.

		Пусть $H \neq (h)$(если совпадают, всё доказано). Тогда $\exists g \in H\setminus(h)$. $g = qh + r$; $q, r \in \mathbb{Z}$, $0 \le r < \abs{h}$.

		Тогда $r = g - qh \in H$, что противоречит с выбором $h$, если $r \neq 0$. Тогда $r = 0$, и $h | g, g \in (h)$. Итак, $H = (h)$.

	  \item $G = \mathbb{Z}_n$. Доказательство ничем, по сути, не отличается. Проходит то же самое рассуждение, если считать $g$ и $h$ целыми числами.

		При этом $h | n$: если $n = qh + r$, то $r = n - qh = -qh \in H$. Если $r \neq 0$, то это противоречит с выбором $h$.
	\end{enumerate}

	Итак, в обоих случаях $H = (h)$, при этом во втором случае, если $h \neq 0$, то $h | n$. 
  \end{proof}
\end{Th}

\begin{remark*}
  В 2-порождённой группе могут быть подгруппы, порождённые любым количеством элементов.
\end{remark*}

\begin{remark*}
  \emph{Знак перестановки} $(-1)^\sigma = sgn \sigma = (-1)^{n(\sigma)}$, где $N(\sigma)$ --- количество инверсий в $\sigma$.

  Значит, $A_n = \set{\sigma \in S_n: (-1)^{\sigma} = 1} \le S_n$ --- знакопеременная группа
\end{remark*}

\begin{Th}[Кэли]
  Если $G$ --- группа, $\abs{G} = n$, то $G$ изоморфно подгруппе в $S_n$.

  \begin{proof}
	$S_n \cong S(G) = \set{\mbox{биекции из $G$ в $G$}}$. 
	
	Пусть $h \in G$. Определим $\varphi_h: G\to G, \varphi_h(g) = hg$. Это биекции, потому что $hg_1 = hg_2 \Rightarrow 
	h^{-1}hg_1 = h^{-1}hg_2 \Rightarrow g_1 = g_2$.

	Итак, $\varphi_h \in S(G)$. 
	
	Пусть $h_1, h_2 \in G$. $\varphi_{h_1,h_2}(g) = h_1h_2g = \varphi_{h_1}(\varphi_{h_2}(g)) = \varphi_{h_1} \circ \varphi_{h_2} (g)$.

	Кроме того, $\varphi_e(g) = g$.
	
	Итак, $H = \set{\varphi_h: h \in G}$ --- подгруппа в $S(G)$: $\varphi_{h_1}\circ\varphi_{h_2} = \varphi_{h_1,h_2} \in H$;

	$\varphi_h \circ \varphi_{h^{-1}} = \varphi_e = id = \varphi_{h^{-1}} \circ \varphi_h \Rightarrow
	(\varphi_h)^{-1} = \varphi_{h^{-1}} \in H$.

	Наконец, $G \cong H$: $\pi: G \to H, \pi(h) = \varphi_h$ --- изоморфизм.

	Сохранение операций: $\pi(h_1, h_2) = \varphi_{h_1,h_2} = \varphi_{h_1} \circ \varphi_{h_2} = \pi(h_1) \circ \pi_{h_2}$. 
	Нетрудно показать также, что $\pi$ --- биекция.
  \end{proof}
\end{Th}

Пусть $\sigma \in S_n$. Стандартная запись --- табличная: $\sigma = ({1 \atop \sigma(1)} {2 \atop \sigma(2)} \cdots {n \atop \sigma(n)})$.

\begin{define*}
  Пусть $a_1, \ldots, a_n \in [n]$ --- различные элементы. Тогда \emph{циклом} $a_1, \ldots, a_k$ называется перестановка $\sigma:
  \sigma(a_i) = a_{i+1}, 1 \le i \le k-1, \sigma(a_k) = a_1, \sigma(b) = b, b > k$. Число $k$ называется \emph{длиной} этого цикла.
  Цикл длины 2 называется\emph{ транспозицией}.

  Семейство циклов $\sigma_1, \ldots, \sigma_t$\emph{ независимое}, если $\forall a \in [n] \to $ есть не более одного цикла
  в семействе: $\sigma(a) \neq a$.
\end{define*}

\begin{example} $ $

  $(12) (346) (57)$ --- независимое.

  $(12) (345) (57)$ --- зависимое.
\end{example}

\begin{claim}$ $
  \begin{enumerate}
	\item Любая перестановка $\sigma \in S_n$ раскладывается в произведение независимых циклов (оно может состоять из одного или нуля элементов,
	  если она сама цикл или тождественная соответственно).

	\item Если $\sigma_1, \sigma_2$ --- независимые циклы, то $\sigma_1\sigma_2 = \sigma_2\sigma_1$.
  \end{enumerate}

  \begin{proof} $ $

	\begin{enumerate}
	  \item Нарисуем орграф перестановки с множеством вершин $[n]$ и рёбрами $k \to \sigma(k)$.

		$\deg_-k = \deg_+k = 1$ ($\sigma^{-1}(k)$). Это значит, что граф разбивается в несвязное объединение ориентированных циклов.

		Эти циклы в графе соответствуют независимым циклам в группе $S_n$.

	  \item Если $\sigma_1, \sigma_2$ --- независимые циклы, то $\forall k \in [n] \to \sigma_1(k) = k \lor \sigma_2(k) = k$.
		Пусть $\sigma_1(k) = k$.

		$\sigma_2\circ \sigma_1(k) = \sigma_2(\sigma_1(k)) = \sigma_2(k)$. 

		Если $\sigma_2(k) \neq =k$, то $\sigma_2(\sigma_2(k)) \neq \sigma_2(k)$. Так как циклы независимы, 
		это значит, что $\sigma_1 \circ \sigma_2(k) = \sigma_1(\sigma_2(k)) = \sigma_2(k)$

		Если же $\sigma_2(k) = k$, то $\sigma_1 \circ \sigma_2(k) = k = \sigma_2 \circ \sigma_1(k)$.

		Итак, $\sigma_1 \circ \sigma_2 (k) = \sigma_2 \circ \sigma_1 (k)$
	\end{enumerate}
  \end{proof}
\end{claim}

\begin{claim}
  Если $\sigma_1, \ldots, \sigma_n$ --- независимые циклы, то $(\sigma_1, \ldots, \sigma_k)^k = \sigma_1^k \circ \ldots \circ \sigma_k^k$.
\end{claim}

\begin{claim}
  Если $\sigma$ -- произведение независимых циклов длин $l_1, \ldots, l_t$, то $\abs{\sigma} = \NOK(l_1, \ldots, l_k)$.

  \begin{proof}
	$\ord(a_1, \ldots, a_k) = k$. Тогда $\sigma = \tau_1 \cdots \tau_t$, то $\sigma^{\NOK(l_1, \ldots, l_t)} = 
	\tau_1^{\NOK(\ldots)} \cdots \tau_t^{\NOK(\ldots)}$.

	Если же $0 < N < \NOK(l_1, \cdots, l_k)$, то $N \not: l_i$. Тогда $\sigma^N = \tau_1^N \cdots \tau_i^N \cdots \tau_t^N$ --- не тождественное преобразование.

	Тогда и $\sigma$ --- не тождественное.
  \end{proof}
\end{claim}

\part*{Лекция 3}

$\lhd \rhd \unlhd \unrhd$

\end{document}

