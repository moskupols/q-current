\documentclass[oneside,russian, a4paper]{amsart}
\usepackage[T2A]{fontenc}
\usepackage[utf8]{inputenc}
\usepackage{fancyhdr}
\usepackage[margin=1in]{geometry}
\pagestyle{fancy}
%\setlength{\parskip}{\medskipamount}
%\setlength{\parindent}{0pt}
\usepackage[russian]{babel}
\usepackage{amsthm}
\usepackage{amstext}
\usepackage{amssymb}
\usepackage{cmap}
\usepackage[unicode=true,
 bookmarks=true,bookmarksnumbered=false,bookmarksopen=false,
 breaklinks=false,pdfborder={0 0 1},backref=false,colorlinks=false]
 {hyperref}
\usepackage{braket} % for \braket, \set and \Set

\makeatletter

\DeclareRobustCommand{\cyrtext}{
  \fontencoding{T2A}\selectfont\def\encodingdefault{T2A}}
\DeclareRobustCommand{\textcyr}[1]{\leavevmode{\cyrtext #1}}
\AtBeginDocument{\DeclareFontEncoding{T2A}{}{}}

%%%%%%%%%%%%%%%%%%%%%%%%%%%%%% Textclass specific LaTeX commands.
%\numberwithin{equation}{section}
%\numberwithin{figure}{section}
 \theoremstyle{definition}
 \newtheorem*{define*}{\protect\definitionname}
 \theoremstyle{plain}
 \newtheorem{Th}{\protect\theoremname}
 \theoremstyle{remark}
 \newtheorem{remark}{\protect\remarkname}
 \theoremstyle{remark}
 \newtheorem*{remark*}{\protect\remarkname}
 \theoremstyle{definition}
 \newtheorem{exercise}{\protect\exercisename}
 \theoremstyle{remark}
 \newtheorem{claim}{\protect\claimname}
 \theoremstyle{remark}
 \newtheorem*{claim*}{\protect\claimname}
 \theoremstyle{definition}
 \newtheorem{example}{\protect\examplename}
 \theoremstyle{plain}
 \newtheorem{lem}{\protect\lemmaname}
 \theoremstyle{plain}
 \newtheorem*{algorithm*}{\protect\algorithmname}

\makeatother
  \providecommand{\claimname}{Утверждение}
  \providecommand{\definitionname}{Определение}
  \providecommand{\examplename}{Пример}
  \providecommand{\exercisename}{Упражнение}
  \providecommand{\lemmaname}{Лемма}
  \providecommand{\remarkname}{Замечание}
  \providecommand{\theoremname}{Теорема}
  \providecommand{\algorithmname}{Алгоритм}

  \renewcommand\le{\leqslant}
  \renewcommand\ge{\geqslant}


\newcommand{\raigorname}[1]{\renewcommand{\raigor}{#1}}
\newcommand{\raigoraction}[1]{\renewcommand{\raigortold}{#1}}
\providecommand\raigor{Райгор}
\providecommand\raigortold{читал}
\providecommand{\gitlink}{Конспект лекций, которые \raigor{} \raigortold{} на ФИВТ МФТИ осенью 2014 года.
Исходники лежат, благодарности полнятся, а багрепорты принимаются по адресу \url{https://github.com/moskupols/q-current}.}

%\newcommand{\relmid}{\mathrel{}|\mathrel{}}
\newcommand{\abs}[1]{\left\lvert #1 \right\rvert}



\begin{document}

\title{Group theory}
\author{Фёдор Алексеев}
\raigorname{И.\,И.\,Богданов}

\maketitle
\gitlink{}
\tableofcontents\newpage{}

\part{Лекция 1. Группа, подгруппа. Изоморфизм групп.}
\section{Группа}
\begin{define*}
  \emph{Группа}~--- это множество $G$ с определённой на нём бинарной операцией (т.\,е. $\forall a, b \in G$ определён результат операции $a\cdot b \in G$)
  \begin{description}
	\item[1. Ассоциативность] $\forall a, b, c \in G \to (a\cdot b)\cdot c = a \cdot (b \cdot c)$;
	\item[2. Наличие нейтрального элемента] $\exists e \in G: \forall a \in G \to a\cdot e = e\cdot a$;
	\item[3. Наличие обратного элемента] $\forall g \in G \to \exists g^{-1} \in G: g\cdot g^{-1} = g^{-1}\cdot g = e$.
  \end{description}
  Группа \emph{абелева}(\emph{коммутативная}), если $\forall a, b \in G \to a\cdot b = b\cdot a$.
\end{define*}

\begin{remark*}
  Произведение $g_1 \cdot g_2 \cdot \cdots \cdot g_n$ также не зависит от порядка.
\end{remark*}

\begin{claim}
  $\forall g \in G \to \exists! g^{-1}$.
\end{claim}

\begin{proof}
  Пусть $a$~--- левый обратный к $g$ (т.\,е. $ag = e$),
  $b$~--- правый обратный к $g$ (т.\,е. $gb = e$). Докажем, что $a=b$:
  \begin{equation*}
	b = eb = (ag)b = a(gb) = ae = a
  \end{equation*}
\end{proof}

\begin{exercise}
  Условие 3 из определения можно заменить на $\forall g \in G \to \exists g^{-1}$~--- правый обратный.
\end{exercise}

\begin{example}
	$(\mathbb{Z}, +)$
\end{example}

\begin{example}
  $(\mathbb{R}, +)$, и даже $(F, +)$.
\end{example}

\begin{example}
  $(\mathbb{R}\backslash\{0\}, 0)$, и даже $F^* = (F\backslash\{0\}, \cdot)$.
\end{example}

\begin{remark*}
  Более общо, если $R$~--- кольцо, то $(R, +)$ и $(R^*, \cdot)$ --- абелевы группы ($R^*$ --- множество обратимых элементов $R$).
  Хорошо бы доказать, впрочем, что если $a, b \in R^*$, то $ab \in R^*$:
  \begin{equation*}
	ab(ab)^{-1} = abb^{-1}a^{-1} = aa^{-1} = e
  \end{equation*}
\end{remark*}

\begin{example}
  $M_{n\times n}(F)$ --- кольцо. $(M_{n\times n})^* = GL(F)$ --- мультипликативная группа невырожденных матриц $n\times n$.
\end{example}

\begin{example}
  $(\mathbb{Z}_n, +)$ и $(\mathbb{Z}_n^*, \cdot)$, где $\mathbb{Z}_n^*$ --- все взаимно простые с $n$.
\end{example}

\begin{proof}
  Если $(a, n) \neq 1$, то $ka \not\equiv 1 \pmod{n}$. Если $(a, n) = 1$, то $\exists u, v: au + nv = a	\Rightarrow	au \equiv 1 \pmod{n}$
\end{proof}
Значит $|\mathbb{Z}_n^*| = \varphi(n)$.

\begin{define*}
  \emph{Порядок группы} G --- это количество её элементов $|G|$.
\end{define*}

\begin{example}
  Пусть $S_n$ --- все перестановки множества $[n] = \left\{ 1, \ldots, n \right\}$, т.\,е. биекции $[n] \to [n]$. Тогда $(S_n, \circ)$ --- группа.

  Если $\Omega$ --- произвольное множество, то аналогичное множество будем обозначать $S(\Omega)$. Это тоже группа.
\end{example}

\section{Подгруппы. Изоморфизмы}

\begin{define*}
  Пусть $G$ --- группа, $\varnothing \neq H \subset G$. $H$ --- \emph{подгруппа} $G$, если $\forall a, b \in H \to ab \in H, a^{-1} \in H$. 
  Обозначается как $H \le G$.
\end{define*}

\begin{define*}
  $\left\{ e \right\} \le G$, $G \le G$ --- несобственные подгруппы.
\end{define*}

\begin{example}
  $D_n(F^*) \le GL_n(F)$, где $D_n(F)$ --- множество диагональных матриц над $F$.
\end{example}

\begin{example}
  $GL_n(F) \ge SL_N(F) = \left\{ A \in GL_n(F) | \det{A} = 1 \right\}$.
\end{example}

\begin{exercise}
  $GL_n(F) \ge T_n(F)$ ($T$ --- верхнетреугольные. Здесь с не нулями на диагонали.).
\end{exercise}

\begin{example}
  $GL_n(R)	\ge O_n$ --- группа ортогональных матриц ($A^{-1} = A^T$).
\end{example}

\begin{example}
  $O_n \ge \mathcal{D}_n = \left\{ f \in O_n: f(P_n) = P_n \right\}$ ($P_n$ --- это многоугольник) --- группа диэдра.
\end{example}

\begin{exercise}
  $|\mathcal{D}_n| = 2n$ (по $n$ поворотов и симметрий).
\end{exercise}

\begin{define*}
  Пусть $G$ и $H$ --- две группы. $\varphi: G\to H$ --- \emph{изоморфизм}, если $\varphi$ --- биекция, и $\forall a, b \to$
  \begin{enumerate}
	\item $\varphi(ab) = \varphi(a)\varphi(b)$
	\item $\varphi(a^{-1}) = (\varphi(a))^{-1}$.
  \end{enumerate}
\end{define*}

\begin{define*}
  Две группы \emph{изоморфны}, если $\exists \varphi$ --- изоморфизм.
\end{define*}

\begin{example}
  $\mathcal{D}_3 \cong S_3$
\end{example}

\begin{example}
  $\mathbb{C}^* \ge \mathbb{C}_n = \left\{ z \in \mathbb{C}: z^n = 1 \right\}$. 
  $\mathbb{Z}_n \cong \mathbb{C}_n \cong$ группе вращений правильного $n$-угольника.
\end{example}

\begin{exercise}
  Свойство 2 определения изоморфизма не нужно.
\end{exercise}

\begin{define*}
  $G$ --- группа, $M \subset G$ --- подмножество. \emph{Подгруппой}, порождённой множеством $M$ называется пересечение всех подгрупп в $G$ содержащих $M$:
  $$(M) = \bigcap_{H\le G, M\le H}H$$
\end{define*}

\begin{claim}
  Пересечение любого семейства подгрупп --- подгруппа. (в т.\,ч., $(M)$ --- подгруппа)
\end{claim}

\begin{proof}
  Пусть $K = \bigcap_{H_i \le G, i \in I}H_i$. Если $a, b \in K$, то 
  $a, b \in H_i \Rightarrow 
  \left\{ 
	\begin{aligned}
	  ab \in H_i &\Rightarrow& ab \in K\\ 
	  a^{-1} \in H_i &\Rightarrow& a^{-1} \in K
	\end{aligned}
  \right.$, кроме того, $e \in k$.
\end{proof}

\begin{claim}
  $(M) = \left\{ a_1, \ldots, a_k: \forall i \le k \to a_i \in M \lor a^{-1} \in M \right\}$.
\end{claim}

\begin{proof}
  Обозначим правую часть за $N$.

  \begin{itemize}
	\item $N \subset (M)$: если $a_i, \ldots, a_k \in N$, то $a_i \in (M) \Rightarrow a_1, \ldots, a_k \in (M)$.
	\item $(M) \subset N$: это так, ибо $N$ --- подгруппа, содержащая $M$.
  \end{itemize}
\end{proof}

\end{document}

