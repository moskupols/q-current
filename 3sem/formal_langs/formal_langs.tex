\documentclass[russian,twoside,a5paper]{amsart}
\usepackage{cmap}
\usepackage[T2A]{fontenc}
\usepackage[utf8]{inputenc}
\usepackage[margin=0.6in]{geometry}
\usepackage{amsthm}
\usepackage{amstext}
\usepackage{amssymb}
\usepackage[unicode=true,
 bookmarks=true,bookmarksnumbered=false,bookmarksopen=false,
 breaklinks=false,pdfborder={0 0 1},backref=false,colorlinks=false]
 {hyperref}
\usepackage{braket} % for \braket, \set and \Set
\usepackage[russian]{babel}
\usepackage{mathtext}

\usepackage{fancyhdr}
\setlength{\headheight}{15pt}

\pagestyle{fancy}

\fancyhf{}
\fancyhead[LE,RO]{\thepage}
\fancyhead[RE]{\textit{\nouppercase{\leftmark}}}
\fancyhead[LO]{\textit{\nouppercase{\rightmark}}}


\makeatletter

\DeclareRobustCommand{\cyrtext}{
  \fontencoding{T2A}\selectfont\def\encodingdefault{T2A}}
\DeclareRobustCommand{\textcyr}[1]{\leavevmode{\cyrtext #1}}
\AtBeginDocument{\DeclareFontEncoding{T2A}{}{}}

%%%%%%%%%%%%%%%%%%%%%%%%%%%%%% Textclass specific LaTeX commands.
%\numberwithin{equation}{section}
%\numberwithin{figure}{section}
 \theoremstyle{definition}
 \newtheorem*{define*}{\protect\definitionname}
 \theoremstyle{plain}
 \newtheorem{Th}{\protect\theoremname}
 \theoremstyle{remark}
 \newtheorem{remark}{\protect\remarkname}
 \theoremstyle{remark}
 \newtheorem*{remark*}{\protect\remarkname}
 \theoremstyle{definition}
 \newtheorem{exercise}{\protect\exercisename}
 \theoremstyle{remark}
 \newtheorem{claim}{\protect\claimname}
 \theoremstyle{remark}
 \newtheorem*{claim*}{\protect\claimname}
 \theoremstyle{definition}
 \newtheorem{example}{\protect\examplename}
 \theoremstyle{plain}
 \newtheorem{lem}{\protect\lemmaname}
 \theoremstyle{plain}
 \newtheorem*{algorithm*}{\protect\algorithmname}

\makeatother
  \providecommand{\claimname}{Утверждение}
  \providecommand{\definitionname}{Определение}
  \providecommand{\examplename}{Пример}
  \providecommand{\exercisename}{Упражнение}
  \providecommand{\lemmaname}{Лемма}
  \providecommand{\remarkname}{Замечание}
  \providecommand{\theoremname}{Теорема}
  \providecommand{\algorithmname}{Алгоритм}

  \renewcommand\le{\leqslant}
  \renewcommand\ge{\geqslant}


\newcommand{\raigorname}[1]{\renewcommand{\raigor}{#1}}
\newcommand{\raigoraction}[1]{\renewcommand{\raigortold}{#1}}
\providecommand\raigor{Райгор}
\providecommand\raigortold{читал}
\providecommand{\gitlink}{Конспект лекций, которые \raigor{} \raigortold{} на ФИВТ МФТИ весной 2015 года.
Исходники лежат, благодарности полнятся, а багрепорты принимаются по адресу \url{https://github.com/moskupols/q-current}.}

%\newcommand{\relmid}{\mathrel{}|\mathrel{}}
\newcommand{\abs}[1]{\left\lvert #1 \right\rvert}

\DeclareMathOperator{\ord}{ord}



\newenvironment{lyxcode}
{\par\begin{list}{}{
\setlength{\rightmargin}{\leftmargin}
\setlength{\listparindent}{0pt}% needed for AMS classes
\raggedright
\setlength{\itemsep}{0pt}
\setlength{\parsep}{0pt}
\normalfont\ttfamily}%
 \item[]}
{\end{list}}

\begin{document}

\title{$\mbox{Теоррия формальных языков}$}
\author{Фёдор Алексеев}
\raigorname{А.\,А.\,Сорокин}

\maketitle
\gitlink\tableofcontents\newpage{}

\part*{Лекция 1. Конечные автоматы}

\section{НКА}
\begin{define*}
\emph{Алфавит} --- непустое конечное множество символов $\Sigma$.
Множество всех слов над данным алфавитом --- $\Sigma^{\star}$.
$\varepsilon\in\Sigma^{\star}$ --- \emph{пустое слово}. $L\subset\Sigma^{\star}$
--- \emph{формальный язык}.

Операция \emph{конкатенации} языков: $L_{1}\cdot L_{2}=\Set{ uv|u\in L_{1},v\in L_{2}} $
(точку будем опускать).
\end{define*}

\begin{define*}
\emph{Конечный недетерминированный автомат }$M=\Braket{ Q,\Sigma,\Delta,q_{0},F} $,
где $Q$ --- множество \emph{состояний},$\Sigma$ --- \emph{алфавит},
$\Delta\subset Q\times\Sigma^{\star}\times Q$ --- конечное множество
\emph{переходов}, $q_{0}\in Q$ --- \emph{стартовое состояние}, $F\subset Q$
--- множество \emph{терминальных состояний}.
\end{define*}

\begin{define*}
Определим $\vdash$ --- наименьшее рефлексивное транзитивное отношение,
такое что

1) $\forall\left(\Braket{ q_{1},w} \mapsto q_{2}\right)\in\Delta\rightarrow\Braket{ q_{1},w} \vdash\Braket{ q_{2},\varepsilon} $

2) $\Braket{ q_{1},u} \vdash\Braket{ q_{2},\varepsilon} $,
$\Braket{ q_{2},w} \vdash\Braket{ q_{3},\varepsilon} \Rightarrow\Braket{ q_{1},uw} \vdash\Braket{ q_{3},\varepsilon} $

3) $\Braket{ q_{1},u} \vdash\Braket{ q_{2},\varepsilon} \Rightarrow\Braket{ q_{1},uv} \vdash\Braket{ q_{2},v} $
\end{define*}

\begin{define*}
Язык задаваемый автоматом $M$ --- $L(M)=\Set{ w\in\Sigma|\exists q\in F:\ \Braket{ q_{0},w} \vdash\Braket{ q,\varepsilon} } $.
Такие языки называются \emph{автоматными}.\end{define*}
\begin{lem}
$\forall L$ --- автоматного языка $\exists M^{\prime}$ --- конечный
автомат, такой что $L(M^{\prime})=L$ и $\forall\left(\Braket{ q_{1},w} \mapsto q_{2}\right)\in\Delta^{\prime}\rightarrow\abs{w}\le1$.\end{lem}
\begin{proof}
Построим автомат явно, добавив на каждом переходе $\abs{w}-1$
новых состояний и сделав между ними однобуквенные переходы.\end{proof}
\begin{lem}
$\forall L$ --- автоматного языка $\exists M^{\prime}$ --- конечный
автомат, такой что все переходы имеют длину $\le1$ и $\abs{F}=1$\end{lem}
\begin{proof}
Строим автомат $M^{\prime}=\Braket{ Q\cup\left\{ q_{f}\right\} ,\Sigma,\Delta^{\prime},q_{0},\left\{ q_{f}\right\} } $,
где $\Delta^{\prime}=\Delta\cup\Set{ \Braket{ q,\varepsilon} \mapsto q_{f}|q\in F} $.\end{proof}
\begin{lem}
$\forall L$ --- автоматного языка $\exists M^{\prime}$ --- НКА с
однобуквенными переходами, его распознающий\end{lem}
\begin{proof}
Построим автомат $M^{\prime}=\Braket{ Q,\Sigma,\Delta^{\prime},q_{0},F^{\prime}} $.

$\Delta^{\prime}=\Delta\cup\Set{ \left(q_{1},a\right)\mapsto q_{2}|\exists q_{3}:\ \left(q_{1},\varepsilon\right)\vdash q_{3},\ \left(q_{3},a\right)\vdash q_{2}} $.
$F^{\prime}=\Set{ q|\exists q^{\prime}\in F^{\prime}:\ \left(\Braket{ q,\varepsilon} \vdash\Braket{ q^{\prime},\varepsilon} \right)} \cup F$.

Пусть $w\in L(M)\Leftrightarrow\exists q\in F\ \exists q_{1},\ldots,q_{n}\in Q:\ \Braket{ q_{0},a_{1},\ldots,a_{n}} \vdash\Braket{ q_{1},a_{1},\ldots,a_{n}} \vdash\ldots\vdash\Braket{ q_{n},\varepsilon} \vdash\Braket{ q,\varepsilon} $.
Тогда в новом автомате это слово будет принято, так как по построению
из состояния $q_{i}$ есть переход по символу $a_{i+1}$ в $q_{i+1}$,
а $q_{n}\in F^{\prime}$.\end{proof}
\begin{algorithm*}[Распознавания слова автоматом]\end{algorithm*}
\begin{lyxcode}
  curQ~=~$\Set{ q_{0}} $

  for~c~in~word:
	\begin{lyxcode}
	newQ~=~$\varnothing$

	for~state~in~curQ:
	  \begin{lyxcode}
	  for~$\left(state,c,newState\right)$~in~$\Delta$:
		\begin{lyxcode}
		  newQ~=~$\mathtt{newQ}\cup\Set{newState} $
		\end{lyxcode}
	  \end{lyxcode}
	curQ~=~newQ
	\end{lyxcode}
  if~$\mathtt{curQ} \cap F=\varnothing$~then~accept~word

  else~discard~word
\end{lyxcode}

\section{ДКА}
\begin{define*}
$M=\Braket{ Q,\Sigma,\Delta,q_{0},F} $ --- ДКА, если

1) $\forall\left(\Braket{ q_{1},a} \mapsto q_{2}\right)\in\Delta\Rightarrow\abs{a}=1$

2) $\forall q\in Q,\ a\in\Sigma\rightarrow\abs{\Set{ q_{2}|\Braket{ q_{1},a} \mapsto q_{2}} }\le1$\end{define*}
\begin{claim*}
$\forall q_{1}\in Q\ \forall w\in\Sigma^{\star}\rightarrow\abs{\set{ q_{2}|\Braket{ q_{1},w} \vdash\Braket{ q_{2},\varepsilon} } }\le1$.\end{claim*}


\part*{Лекция 2. }

\begin{Th}
  Для любого языка $L$, распознаваемого некоторым НКА, существует ДКА $M$: $\mathcal{L} = \mathcal{L}(M)$.

  \begin{proof}
	Пусть $\mathcal{L} = \mathcal{L}(M_0), M_0 = \braket{Q_0, \Sigma, \Delta_0, \bar{q_0}, F_0}$ --- НКА с однобуквенными переходами.

	$Q = 2^{Q_0}, q_0 = {\bar{q_0}}, F = \set{Q' \subseteq Q | Q' \cap F_0 \neq \varnothing}$.

	$M = \braket{Q, \Sigma, \Delta_0, q_0, F}$

	$\Delta(Q', a) = \set{q | \exists q' \in Q' : \braket{q', a, q} \in \Delta_0}$

	$\Delta_M(q, w)$ --- состояние, в которое переходит $M$ по $w$.
	\begin{lem}
	  $\Delta_M(q_0, w) = \set{q \in Q_0 | \braket{\bar{q_0}, w} \vdash_{M_0} \braket{q, \varepsilon}}$
	  \begin{proof}
		Индукция по $w$.

		База: $w = \varepsilon$. $\Delta_M(q_0m \varepsilon) = q_0 = \set{\bar{q_0}}$

		Шаг: $w = ua, a \in \Sigma$. $\Delta_M(\Delta_M(q_0, u), a) = 
		\Delta_M(\set{\tilda{q} \in Q_0 | \braket{\bar{q_0}, 1} \vdash_{M_0} \braket{\tilda{q}, \varepsilon}}, a) = 
		\set{q | \exists \tilda{q} : \braket{\bar{q_0}, u} \vdash \braket{\tilda{q}, \varepsilon} \wedge \braket{\tilda{q}, u, q} \in \Delta_{M_0}} = 
		\set{q | \braket{\bar{q_0}, ua} \vdash \braket{q, \varepsilon}}$
	  \end{proof}
	\end{lem}

	$\mathcal{L}(M) = \set{w | \exists q \in F : \braket{q_0, w} \vdash_M \braket{q, \varepsilon}} = \set{w | \Delta_M(q_0, w) \in F} =
	\set{w | \set{\tilda{q} \in Q_0 | \braket{\bar{q_0}, w} \vdash \braket{\tilda{q}, \varepsilon}} \in F} =
	\set{w | \exists \tilda{q} \in Q_0 : \tilda{q} \in F_0 \wedge \braket{\bar{q_0}, w} \vdash \braket{\tilda{q}, \varepsilon}} = \mathcal{L}(M_0)$
  \end{proof}
\end{Th}

\begin{Th}
  Если $L_1, L_2$ --- автоматные языки, то
  \begin{enumerate}
	\item $L_1L_2$ --- автоматный.
	\item $L_1 \cup L_2$ --- автоматный.
	\item $L_1 \cap L_2$ --- автоматный.
	\item $L_1^*$ --- автоматный.
	\item $\Sigma^* - L_1$ --- автоматный.
  \end{enumerate}

  \begin{proof}
	\begin{enumerate}
	  \item $L_1L_2 = \mathcal{L}(M), M = \braket{Q_1 \cup Q_2, \Sigma, \Delta, q_{01}, F_2}.$

		$\Delta = \Delta_1 \cup \Delta_2 \cup set{\braket{q, \varepsilon} \to q_{02} | q \in F_1}$.

	  \item $L_1 \cup  L_2 = \set{\set{q_0} \cup Q_1 \cup Q_2, \Sigma, \Delta, q_0, F_1 \cup F_2}$

		$\Delta = \Delta_1 \cup \Delta_2 \cup set{\braket{q_0, \varepsilon} \to q_{01}, \braket{q_0, \varepsilon} \to q_{02}}$

	  \item $L_1 \cap L_2 = \set{Q_1 \times Q_2, \Sigma, \Delta, \braket{q_{01}, q_{02}}, F_1 \times F_2}$

		$\Delta = \Set{\braket{\braket{q_1, q_2}, a}} \to \braket{q'_1, q'_2} | 
		\begin{aligned}

		\end{aligned}$

	  \item 

	  \item $ $
		\begin{define*}
		  $M = \braket{Q, \Sigma, \Delta, q_0, F}$ --- полный ДКА, если $M$ --- ДКА и $\Delta: Q \times \Sigma \to Q$ всюду определена.
		\end{define*}
		\begin{claim}
		  Любой автоматный язык распознаётся полным ДКА.
		  \begin{proof}
			$L = L(M_0), M_0 = \braket{Q, \Sigma, \Delta, q_0, F}$ --- ДКА.

			$L = L(M), M = \braket{Q \cup \set{q_1}, \Sigma, \Delta', q_1, F}$

			$\Delta' = \Delta \cup \set{\braket{q, a} \to q_1 | \lnot (\exists q' : \braket{q, a} \to q' \in A)} \cup
			\set{}$
		  \end{proof}
		\end{claim}
	\end{enumerate}
  \end{proof}
\end{Th}

\section*{Регулярные выражения}

\begin{define*}
  Множеством регулярных выражений над алфавитом $\Sigma$ $Reg(\Sigma)$:
  \begin{enumerate}
	\item $\Sigma \subset Reg(\Sigma))$
	\item $0, 1 \in Reg(\Sigma)$
	\item $\forall \alpha, \beta \in Reg(\Sigma) \to \alpha\beta, \alpha + \beta, a^* \in Reg(\Sigma)$.
  \end{enumerate}
\end{define*}

\begin{define*}
  Язык\ldots
\end{define*}

\begin{Th}
  Множетва регулярных и автоматных языков совпадают.

  \begin{proof}
	То, что всякий регулярный язык автоматный, доказывается индукцией по построению регулярного выражения, используя предыдущую теорему.

	Докажем, что всякий автоматный язык регулярен.
	\begin{define*}[неформальное]
	  Расширенный автомат --- автомат с регулярными выражениями в качестве меток перехода, такой что между любыми двумя состояниями не более 1 ребра
	  (этого всегда можно добиться, используя регурярное выражение). 
	\end{define*}

	\begin{remark*}
	  Тогда обычный автомат --- частный случай расширенного.
	\end{remark*}

	\begin{lem}
	  Всякий язык, задаваемый расширенным автоматом, регулярен.
	  \begin{proof}
		Индукция по $|Q|$(числуу состояний).

		База: $|Q| = 1$. $|Q| = 2$ --- показывается регулярка, задаваемая автоматом.

		Переход: Идея в том, чтобы умееньшить число состояний. Можно считать, что завершающих состояний ровно одно. 
		Тогда $\exists q$ --- не стартовое не завершающее состояние. Рассмотри все пары $q_1, q_2$: $1q = \alpha, q2 = \beta, qq = \gamma$.
		Теперь добавим переходы $12 = \alpha\gamma^*\beta$. Теперь можно удалить $q$.
	  \end{proof}
	\end{lem}
  \end{proof}
\end{Th}

\begin{Th}
  Для любого автоматного языка $L \exists p \in \mathbb{N} \forall w \in L \abs{w} \ge p \to 
  \exists x, y, z : \abs{xy} \le p, \abs{y} > 0, w = xyz, \forall k \in \mathbb{N} xy^kz \in L$.
  \begin{proof}
	$L = L(M), M = \braket{Q, \Sigma, \Delta, q_0, F}$ --- НКА с однобуквенными переходами.
  \end{proof}<++>
\end{Th}<++>

\end{document}

