\documentclass[oneside,russian, a4paper]{amsart}
\usepackage[T2A]{fontenc}
\usepackage[utf8]{inputenc}
\usepackage{fancyhdr}
\usepackage[margin=1in]{geometry}
\pagestyle{fancy}
%\setlength{\parskip}{\medskipamount}
%\setlength{\parindent}{0pt}
\usepackage[russian]{babel}
\usepackage{amsthm}
\usepackage{amstext}
\usepackage{amssymb}
\usepackage{cmap}
\usepackage[unicode=true,
 bookmarks=true,bookmarksnumbered=false,bookmarksopen=false,
 breaklinks=false,pdfborder={0 0 1},backref=false,colorlinks=false]
 {hyperref}
\usepackage{braket} % for \braket, \set and \Set

\makeatletter

\DeclareRobustCommand{\cyrtext}{
  \fontencoding{T2A}\selectfont\def\encodingdefault{T2A}}
\DeclareRobustCommand{\textcyr}[1]{\leavevmode{\cyrtext #1}}
\AtBeginDocument{\DeclareFontEncoding{T2A}{}{}}

%%%%%%%%%%%%%%%%%%%%%%%%%%%%%% Textclass specific LaTeX commands.
%\numberwithin{equation}{section}
%\numberwithin{figure}{section}
 \theoremstyle{definition}
 \newtheorem*{define*}{\protect\definitionname}
 \theoremstyle{plain}
 \newtheorem{Th}{\protect\theoremname}
 \theoremstyle{remark}
 \newtheorem{remark}{\protect\remarkname}
 \theoremstyle{remark}
 \newtheorem*{remark*}{\protect\remarkname}
 \theoremstyle{definition}
 \newtheorem{exercise}{\protect\exercisename}
 \theoremstyle{remark}
 \newtheorem{claim}{\protect\claimname}
 \theoremstyle{remark}
 \newtheorem*{claim*}{\protect\claimname}
 \theoremstyle{definition}
 \newtheorem{example}{\protect\examplename}
 \theoremstyle{plain}
 \newtheorem{lem}{\protect\lemmaname}
 \theoremstyle{plain}
 \newtheorem*{algorithm*}{\protect\algorithmname}

\makeatother
  \providecommand{\claimname}{Утверждение}
  \providecommand{\definitionname}{Определение}
  \providecommand{\examplename}{Пример}
  \providecommand{\exercisename}{Упражнение}
  \providecommand{\lemmaname}{Лемма}
  \providecommand{\remarkname}{Замечание}
  \providecommand{\theoremname}{Теорема}
  \providecommand{\algorithmname}{Алгоритм}

  \renewcommand\le{\leqslant}
  \renewcommand\ge{\geqslant}


\newcommand{\raigorname}[1]{\renewcommand{\raigor}{#1}}
\newcommand{\raigoraction}[1]{\renewcommand{\raigortold}{#1}}
\providecommand\raigor{Райгор}
\providecommand\raigortold{читал}
\providecommand{\gitlink}{Конспект лекций, которые \raigor{} \raigortold{} на ФИВТ МФТИ осенью 2014 года.
Исходники лежат, благодарности полнятся, а багрепорты принимаются по адресу \url{https://github.com/moskupols/q-current}.}

%\newcommand{\relmid}{\mathrel{}|\mathrel{}}
\newcommand{\abs}[1]{\left\lvert #1 \right\rvert}



\begin{document}

\title{Методы оптимизации}
\author{Фёдор Алексеев}
\raigorname{Даниил Владимирович Мусатов}

\maketitle
\gitlink{}
\tableofcontents\newpage{}




\section*{Оргвопросы}
Курс будет посвящён оптимизации функций непрерывного параметра.
Наверняка, вы знаете, там, теорему Ферма, критерии Лагранжа, и так далее.
Будем считать, что вы что-то уже знаете. Я постараюсь дать меньше матана, и больше алгоритмов.

Оценка складывается из нескольких вещей:
\begin{itemize}
  \item[$40\%$] за две контрольные работы, их нельзя переписывать.
    Задания будут по возможности на рассуждения, а не на счёт.
  \item[$30\%$] два домашних счётных задания. Можно писать программы, можно матлаб, главное чётко потом описать, что вы делали.
  \item[$30\%$] индивидуальный проект. Примерно три вида:
    \begin{itemize}
      \item теоретический навроде реферата: разобраться в теореме, например, обзор области.
      \item (теоретико-)программистский: запрогать конкретный метод и, главное, хорошенько проанализировать, потестить. Хорошо бы ещё соптимизировать потом
      \item практический: написать простую программу и посмотреть, как она работает на реальных данных. Как правило, это линейное программирование.
        Ничего сложного в реализации, интереснее анализ.
    \end{itemize}
    Можно предлагать свои темы.
\end{itemize}
На эти проценты можно смотреть, как на вес оценки за очередную часть.

Литература:
\begin{itemize}
  \item Ф.\,П. Васильев, ``Методы оптимизации''
  \item А.\,Ф. Изма{\em и}лов, М.\,В. Солодов, ``Численные методы оптимизации''
\end{itemize}

\section{Задача оптимизации}
\subsection{Какие бывают задачи оптимизации?}

Самая общая: есть функция $f: X \rightarrow R$, и хочется найти точку экстремума и сам экстремум.
Экстремумы бывают строгие и нестрогие, бывает минимум и максимум, бывают глобальные и локальные.

Будем для определённости всегда искать минимум. Основное различие в методах для глобальных и локальных.

Типична ситуация, когда $f$ определена на более широком множестве $Y$, а $X \subset Y$.
Часто $X$ задаётся некоторым набором условий вида
$g_1(x) \le 0, \ldots, g_k(x)\le 0, g_{k+1}(x)=0, \ldots, g_m(x)=0$.

Какие бывают методы оптимизации?

Во первых, можно грубо разделить из на аналитические и численные. Например, метод множеств Лагранжа~--- аналитический метод.
Аналитическим методом не всё реально решить.

Основной подкласс численных методов~--- итеративные методы.
Строится некоторая последовательность, в пределе достигающая искомую точку.
Они уже делятся на точные и приближённые, в зависимости от того, достигаем ли мы точного ответа за конечное число итераций.

\subsection{Линейное программирование}

В двух словах, линейное программирование~--- минимизация на многограннике линейной функции вида
$f~: R^n~\rightarrow~R, f(x) = \sum_{i=1}^n c_ix_i = \braket{c,x}$.

Условия многогранника имеют вид $Ax \le b$, это можно писать в матричном виде или в виде системы линейных неравенств или уравнений.

\begin{example}
  Пусть имеются некоторое множество видов товаров $1,\ldots,n$ и множество ресурсов $1,\ldots,m$.
  И есть запасы ресурсов $w_1,\ldots,w_n$. На производство товара $i$ нужно $c_{ij}$ единиц ресурса $j$.
  И есть цены на товары $p_1,\ldots,p_n$.

  Задача состоит в максимизации прибыли, то есть если $x_1,\ldots,x_n$ --- объёмы производства,
  нужно максимизировать $\sum p_ix_i$ при условии, что $x_i \ge 0$ и $\forall j \to \sum_{i=1}^n x_ic_{ij} \le w_j$.
\end{example}

Как может выглядеть оптимальное решение такой задачи?
Вообще говоря, набор условий вовсе не обязательно задаёт ограниченный многогранник, но можно считать, что у нас скорее полиэдр, то есть многогранное множество.
Впрочем, легко привести пример, когда из-за неограниченности минимума вовсе не существует.
Бывает и такое, что решений несколько или даже бесконечно много, но на практике это нечастый случай.

\begin{remark*}
  Вершина~--- это 0-мерная грань, ребро~--- 1-мерная, и так далее.

  С другой стороны, если, например, если у нас тремя неравенствами задан треугольник, то в его вершинах два из трёх неравенств обращаются в равенство, а на сторонах одно.

  В общем случае в пространстве размерности $n$ грань $k$-той размерности~--- это множество точек, в которых одни и те же $n-k$ неравенств обратились в равенство, 
  а остальные неравенства выполняются.
\end{remark*}

\begin{remark*}
  Впрочем, это довольно грубое определение, так как бывают вырожденные случаи, когда в равенство обращается ещё какое-то неравенство, например, 
  ещё одна прямая проходит через вершину треугольника.
\end{remark*}

\begin{claim}
  Минимум линейной функции всегда достигается на некоторой грани.
\end{claim}
\begin{remark*}
  Значит, он достигается в некоторой вершине.
\end{remark*}

Таким образом, задача линейного программирования~--- выяснить, достигается ли минимум, и, если да, то в какой вершине.

Тривиальный метод: перебор всех вершин.
Ясно, что их конечное число, не больше, чем $C_{m}^{n}$, если неравенств $m$. 
Однако это довольно много, даже у куба уже $2^n$ вершин при $2n$ неравенствах, то есть размер перебора растёт экспоненциально от длины записи задачи.

Симплекс-метод позволяет обходить вершины, уменьшая значение функции с каждой итерацией.
Для практических задач подходит неплохо, однако есть примеры, на которых тоже работает экспоненциально долго.

\end{document}

