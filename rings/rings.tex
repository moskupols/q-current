\documentclass[russian,twoside,a5paper]{amsart}
\usepackage{cmap}
\usepackage[T2A]{fontenc}
\usepackage[utf8]{inputenc}
\usepackage[margin=0.6in]{geometry}
\usepackage{amsthm}
\usepackage{amstext}
\usepackage{amssymb}
\usepackage[unicode=true,
 bookmarks=true,bookmarksnumbered=false,bookmarksopen=false,
 breaklinks=false,pdfborder={0 0 1},backref=false,colorlinks=false]
 {hyperref}
\usepackage{braket} % for \braket, \set and \Set
\usepackage[russian]{babel}
\usepackage{mathtext}

\usepackage{fancyhdr}
\setlength{\headheight}{15pt}

\pagestyle{fancy}

\fancyhf{}
\fancyhead[LE,RO]{\thepage}
\fancyhead[RE]{\textit{\nouppercase{\leftmark}}}
\fancyhead[LO]{\textit{\nouppercase{\rightmark}}}


\makeatletter

\DeclareRobustCommand{\cyrtext}{
  \fontencoding{T2A}\selectfont\def\encodingdefault{T2A}}
\DeclareRobustCommand{\textcyr}[1]{\leavevmode{\cyrtext #1}}
\AtBeginDocument{\DeclareFontEncoding{T2A}{}{}}

%%%%%%%%%%%%%%%%%%%%%%%%%%%%%% Textclass specific LaTeX commands.
%\numberwithin{equation}{section}
%\numberwithin{figure}{section}
 \theoremstyle{definition}
 \newtheorem*{define*}{\protect\definitionname}
 \theoremstyle{plain}
 \newtheorem{Th}{\protect\theoremname}
 \theoremstyle{remark}
 \newtheorem{remark}{\protect\remarkname}
 \theoremstyle{remark}
 \newtheorem*{remark*}{\protect\remarkname}
 \theoremstyle{definition}
 \newtheorem{exercise}{\protect\exercisename}
 \theoremstyle{remark}
 \newtheorem{claim}{\protect\claimname}
 \theoremstyle{remark}
 \newtheorem*{claim*}{\protect\claimname}
 \theoremstyle{definition}
 \newtheorem{example}{\protect\examplename}
 \theoremstyle{plain}
 \newtheorem{lem}{\protect\lemmaname}
 \theoremstyle{plain}
 \newtheorem*{algorithm*}{\protect\algorithmname}

\makeatother
  \providecommand{\claimname}{Утверждение}
  \providecommand{\definitionname}{Определение}
  \providecommand{\examplename}{Пример}
  \providecommand{\exercisename}{Упражнение}
  \providecommand{\lemmaname}{Лемма}
  \providecommand{\remarkname}{Замечание}
  \providecommand{\theoremname}{Теорема}
  \providecommand{\algorithmname}{Алгоритм}

  \renewcommand\le{\leqslant}
  \renewcommand\ge{\geqslant}


\newcommand{\raigorname}[1]{\renewcommand{\raigor}{#1}}
\newcommand{\raigoraction}[1]{\renewcommand{\raigortold}{#1}}
\providecommand\raigor{Райгор}
\providecommand\raigortold{читал}
\providecommand{\gitlink}{Конспект лекций, которые \raigor{} \raigortold{} на ФИВТ МФТИ весной 2015 года.
Исходники лежат, благодарности полнятся, а багрепорты принимаются по адресу \url{https://github.com/moskupols/q-current}.}

%\newcommand{\relmid}{\mathrel{}|\mathrel{}}
\newcommand{\abs}[1]{\left\lvert #1 \right\rvert}

\DeclareMathOperator{\ord}{ord}



\begin{document}

\title{Теория колец и полей}
\author{Фёдор Алексеев}
\raigorname{Дмитрий Геннадьевич Ильинский}

\maketitle
\gitlink{}
\tableofcontents\newpage{}

\section*{Оргвопросы}
\label{sec:organization}
К экзамену будут известны списки вопросов на $2$, на $3$--$4$, $5$--$6$, $7$--$8$ и $9$--$10$.

После прихода на экзамен вам последовательно выдаются листочки с вопросами из разных групп.
На экзамен даётся три часа.

Ещё в какой-то момент будет контрольная, её вопросы тоже будут известны заранее.

Курс из трёх частей:
\begin{itemize}
  \item обобщения основной теоремы арифметики, теория делимости
  \item расширения полей, основная теорема алгебры, конечные поля(, коды БЧХ)
  \item $Q \rightarrow R$, $Q \rightarrow Q_p$
\end{itemize}

\section{First}
\label{sec:first}

\begin{define*}
  $\braket{K, +, \cdot}$~--- кольцо, если 
  \begin{enumerate}
    \item $\braket{K,+}$~--- абелева группа
    \item дистрибутивность:
  \end{enumerate}
\end{define*}

Можно добавлять ещё свойства, например:
\begin{itemize}
  \item ассоциативность: $\forall a,b,c \in K \rightarrow (ab)c = a(bc)$
  \item единица: $\exists 1 \in K : \forall a \in K \rightarrow a\cdot 1 = 1 \cdot a = a$
  \item коммутативность: $\forall a, b \rightarrow ab=ba$
  \item обратимость: $\forall K \ni g\ne 0 \rightarrow \exists b : gb = bg = 1$
\end{itemize}

\begin{define*}
  Коммутативное кольцо~--- ассоциативное коммутативное кольцо с единицей. Буквой $K$ обозначают коммутативные кольца.
\end{define*}

\begin{example}[кольца без единицы]
  Удвоенные целые числа.
\end{example}

\begin{example}[некоммутативного кольца с единицей]
  $M_n(R)$, кольцо матриц.  
\end{example}

\begin{example}
  $\braket{M_n(R), +, [,]}$, $[A,B] = AB-BA$
  
  $[[A,B],C] + [[B,C],A] + [[C,A],B] = 0$
\end{example}

\begin{example}
  $\set{0}$~--- коммутативное кольцо, да. Можно называть его тривиальным.
\end{example}

Чем отличается $Z$ от просто коммутативного кольца? В нём нет делителей нуля.

\begin{define*}
  Область целостности~--- коммутативное кольцо, в котором нет делителей нуля.
\end{define*}

\begin{example}[делителя нуля]
    $2\cdot3 = 0$ в $Z_6$.
\end{example}

Рассмотрим множество $\Set{a+b\cdot z | a,b \in Z, z \in C}$. Это кольцо?

Рассмотрим гауссовы числа: $\Set{a+bi | a,b \in Z} = Z[i]$.

\begin{itemize}
    \item $(a+bi)+(c+di) = (a+c)+ (b+d)i \in Z[i]$
    \item $(a+bi)\cdot(c+di) = (ac-bd) + (ad+bc)i \in Z[i]$
    \item $0 + 0i \in Z[i]$
    \item $1 + 0i \in Z[i]$
    \item $a+bi \in Z[i] \Rightarrow (-a) + (-b)i \in Z[i]$
    \item нет делителей нуля
\end{itemize}

\begin{define*}
  $F$~--- поле, если
  \begin{itemize}
    \item это коммутативное кольцо
      \item $\forall a\ne 0 \in F \rightarrow \exists b \in F : ab = ba = 1$
      \item $1 \ne 0$
  \end{itemize}
\end{define*}

\begin{claim}
  В поле нет делителей нуля.
\end{claim}

\begin{proof}
  Пусть $a$~--- делитель нуля: $\exists b : ab = 0$.

  $$
    0 = ab \Rightarrow 0a^{-1} = (ba)a^{-1} = b(ba^{-1}) = b\cdot 1 = b
  $$
\end{proof}

Пусть $K$~--- область целостности.
$a|b \Leftrightarrow \exists c : ac = b$.

Свойства делимости:

\begin{enumerate}
  \item $a | b, b | c \Rightarrow a | c$
    \item $a | b, a | c \Rightarrow a | (b+c)$
    \item $a | 1 \Rightarrow \exists b : ab = 1 \Rightarrow a$~--- обратимый элемент.
      В этом случае любой элемент $K$ делится на $a$.

      \begin{proof}
        Пусть $x\in K$. Тогда $x = 1\cdot x = a(a^{-1}x) \Rightarrow a|x$
      \end{proof}
\end{enumerate}

\begin{example}
  В $Z$ обратимых элемента два: $1, -1$.
\end{example}

\begin{define*}
  Пусть $K^* \subset K$~--- множество обратимых элементов. Тогда $K^*$ называется подгруппой обратимых элементов.
\end{define*}

\begin{exercise}
  $K^*$~--- группа по умножению.
\end{exercise}

\begin{define*}
  $a \sim b$, или $a$ и $b$ ассоциированы, если $\exists r \in K^* : a = rb$
\end{define*}

\begin{exercise}
  ``$\sim$''~--- отношение эквивалентности.
\end{exercise}

\begin{Th}[основная теорема арифметики]
  Любой элемент поля раскладывается единственным образом в произведение простых
\end{Th}
\begin{proof}
  План доказательства:
  \begin{enumerate}
    \item простые числа делятся на себя и на 1.
    \item любое число раскладывается в произведение простых.
    \item $p$ простое $\Rightarrow ab : p \Longrightarrow a:p \lor b:p$
    \item единственность разложения на простые.
  \end{enumerate}
\end{proof}

\begin{define*}
  Элемент кольца называется неприводимым, или неразложимым, если 
  \begin{enumerate}
      \item $x \not\in K^* \cup \set{0}$.
      \item $x = ab \Longrightarrow a \in K^* \lor b \in K^*$
  \end{enumerate}
\end{define*}

\begin{define*}
  $x \in K$ называется простым, если
  \begin{enumerate}
    \item $x \not\in K^* \cup \set{0}$.
    \item $x | ab \Longrightarrow x | a \lor x | b$.
  \end{enumerate}
\end{define*}

\end{document}

