\documentclass[oneside,russian, draft]{amsart}
\usepackage[T2A]{fontenc}
\usepackage[utf8]{inputenc}
\usepackage{fancyhdr}
\usepackage[margin=1in]{geometry}
\pagestyle{fancy}
%\setlength{\parskip}{\medskipamount}
%\setlength{\parindent}{0pt}
\usepackage[russian]{babel}
\usepackage{amsthm}
\usepackage{amstext}
\usepackage{amssymb}
\usepackage{mathdots}
\usepackage{cmap}
\usepackage[unicode=true,
 bookmarks=true,bookmarksnumbered=false,bookmarksopen=false,
 breaklinks=false,pdfborder={0 0 1},backref=false,colorlinks=false]
 {hyperref}

\makeatletter

\DeclareRobustCommand{\cyrtext}{
  \fontencoding{T2A}\selectfont\def\encodingdefault{T2A}}
\DeclareRobustCommand{\textcyr}[1]{\leavevmode{\cyrtext #1}}
\AtBeginDocument{\DeclareFontEncoding{T2A}{}{}}

%%%%%%%%%%%%%%%%%%%%%%%%%%%%%% Textclass specific LaTeX commands.
%\numberwithin{equation}{section}
%\numberwithin{figure}{section}
 \theoremstyle{definition}
 \newtheorem*{define*}{\protect\definitionname}
 \theoremstyle{plain}
 \newtheorem{Th}{\protect\theoremname}
 \theoremstyle{remark}
 \newtheorem{remark}{\protect\remarkname}
 \theoremstyle{remark}
 \newtheorem*{remark*}{\protect\remarkname}
 \theoremstyle{definition}
 \newtheorem{exercise}{\protect\exercisename}
 \theoremstyle{remark}
 \newtheorem{claim}{\protect\claimname}
 \theoremstyle{remark}
 \newtheorem*{claim*}{\protect\claimname}
 \theoremstyle{definition}
 \newtheorem{example}{\protect\examplename}
 \theoremstyle{plain}
 \newtheorem{lem}{\protect\lemmaname}

\makeatother
  \providecommand{\claimname}{Утверждение}
  \providecommand{\definitionname}{Определение}
  \providecommand{\examplename}{Пример}
  \providecommand{\exercisename}{Упражнение}
  \providecommand{\lemmaname}{Лемма}
  \providecommand{\remarkname}{Замечание}
  \providecommand{\theoremname}{Теорема}

  \renewcommand\le{\leqslant}
  \renewcommand\ge{\geqslant}

\newcommand{\raigorname}[1]{\providecommand{\raigor}{#1}}
\providecommand{\gitlink}{Конспект лекций, которые \raigor{} читал для групп 394-397 и 399 ФИВТ МФТИ осенью 2014 года.
Исходники лежат, а багрепорты принимаются по адресу \url{https://github.com/moskupols/q-current}.}


